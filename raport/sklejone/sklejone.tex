\documentclass[]{mwart}

\usepackage{polski}
\usepackage[utf8]{inputenc}

\usepackage{amsthm}
\usepackage{amsmath}
\usepackage{amssymb}

\usepackage{mdframed}
\usepackage{hyperref}
\usepackage[%draft%      % dla obrazkow zakomentowac draft
]{graphicx}  
\usepackage{url}
\usepackage{enumitem}
\usepackage{verbatim}



\usepackage{caption}   
\usepackage{subcaption}

\usepackage{float}      



\usepackage{fancyhdr}
\pagestyle{fancy}
\fancyhf{}

\fancyhead[L]{\includegraphics[height=0.666cm]{wspolne_dla_wszystkich/logo_projektu.png}}
\fancyhead[C]{\textit{Poprawa jakości zdjęć}}
\fancyhead[R]{\includegraphics[height=0.9cm]{wspolne_dla_wszystkich/logo_uczelni.png}}
\fancyfoot[C]{\thepage}

\setlength{\headheight}{20pt}  



\usepackage{listings}
\usepackage{xcolor} 

\makeatletter
\@addtoreset{section}{part}
\makeatother

\setlength{\headheight}{29.20734pt}



\begin{document}




\thispagestyle{empty}

\begin{figure}[h]
    \centering
    \includegraphics[width=1\textwidth]{wspolne_dla_wszystkich/logo_uczelni.png}
\end{figure}


\begin{center}
    {\LARGE \textbf{Poprawa jakości skanów zdjęć wykonanych techniką analogową
        }} \\[0.3cm]
    {\large \textbf{Raporty I-V}} \\[0.2cm]
    \textit{projekt realizowany pod opieką prof. dr hab. inż. Artura Przelaskowskiego}

\end{center}

\begin{figure}[h]
    \centering
    \includegraphics[width=1\textwidth]{wspolne_dla_wszystkich/logo_projektu.png}
\end{figure}

\vfill
\begin{abstract}
    Zbiór raportów projektu poprawy jakości cyfrowych skanów zdjęć wykonanych techniką
    analogową przez grupę nr 9 (wtorkową z godziny 18)
    w składzie:  Bartosz Wójcik, Katarzyna Szwed, Natalia Szymańska,
    Patrycja Szałajko, Aleksandra Wójcik, Karol Sęk, Michał Juszkiewicz, Filip Sajko.
\end{abstract}


\newpage
\tableofcontents
\newpage





























































\part{}
\thispagestyle{empty}

\begin{figure}[h]
    \centering
    \includegraphics[width=1\textwidth]{wspolne_dla_wszystkich/logo_uczelni.png}
\end{figure}

%    \title{Projekt poprawy jakości zdjęć \\ \large{Raport 1}}
%    \date{\today}
%    %\author{\textit{Pod opieką: prof. dr hab. inż. Artura Przylaskowskiego}}
%    \maketitle
%    \begin{center}
%        \textit{Pod opieką: prof. dr hab. inż. Artura Przylaskowskiego}
%    \end{center}

\begin{center}
    {\LARGE \textbf{Projekt poprawy jakości zdjęć analogowych i cyfrowych o skrajnej jasności}} \\[0.3cm]
    {\large \textbf{Raport I}} \\[0.2cm]
    \textit{projekt realizowany pod opieką prof. dr hab. inż. Artura Przelaskowskiego}

\end{center}

\begin{figure}[h]
    \centering
    \includegraphics[width=1\textwidth]{wspolne_dla_wszystkich/logo_projektu.png}
\end{figure}

\vfill
\begin{abstract}
    Raport 1 projektu poprawy jakości zdjęć wykonanych analogowych przez grupę wtorkową z godziny 18
    w składzie:  Bartosz Wójcik, Katarzyna Szwed, Natalia Szymańska,
    Patrycja Szałajko, Aleksandra Wójcik, Karol Sęk, Michał Juszkiewicz, Filip Sajko.

    W tym raporcie skupimy się na opisie procesu wyboru tematu projektu, jego definiowaniu
    i planach jego analizy i rozwiązania. Ponadto wykonamy zdjęcia i zaczniemy je analizować.
\end{abstract}



\newpage



\section{Wstęp}
\subsection{Wybór problemu}
Podczas wyboru tematu projektu kierowaliśmy się użytecznością
i ponadczasowością naszej pracy oraz tym, żeby nasz program
rozwiązywał problem,  który zauważamy na co dzień. Zebraliśmy
kilka naszych najlepszych pomysłów.

Były to: silnik szachowy, inteligentne
rekomendacje muzyki, wydobywanie detali ze zdjęć o ekstremalnej
jasności, opracowywanie wykresów związków chemicznych.
Dodatkowo przeprowadziliśmy ankietę wśród wybranych 27
studentów naszej uczelni. Zapytaliśmy ich, rozwiązanie
którego z powyższych problemów chcieliby zobaczyć.

Wyniki ankiety przedstawiamy poniżej:


\begin{figure}[H]
    \centering
    \includegraphics[width=1\textwidth]{Photos1/wyniki_ankiety.png}
\end{figure}




Jak widać zdecydowanie największą popularnością
cieszył się pomysł stworzenia programu poprawiającego jakość zdjęć.

\subsection{Cel projektu}
W ramach tego projektu skoncentrujemy się na opracowaniu metod
umożliwiających stworzenie algorytmu, który normalizuje jasność
zdjęć charakteryzujących się niedoświetleniem lub
prześwietleniem. Na podstawie tego algorytmu planujemy opracować
przyjazną użytkownikowi aplikację, dedykowaną poprawie jakości
amatorskich fotografii. Odbiorcami naszego projektu są zarówno
miłośnicy fotografii analogowej, jak i entuzjaści, którzy nie
dysponują zaawansowanym sprzętem fotograficznym, a także
wszyscy ci, którzy chcą wydobyć z archiwów rodzinnych
nową jakość.

Fotografia analogowa, w przeciwieństwie do fotografii cyfrowej,
nie pozwala na weryfikację efektów pracy tuż po wykonaniu zdjęcia.
O błędach technicznych fotograf dowiaduje się dopiero po wywołaniu
kliszy, co nie pozwala na wprowadzanie poprawek na bieżąco.
Kluczowym aspektem wykonania poprawnej fotografii jest odpowiednie
naświetlenie zdjęcia -- aby uniknąć utraty detali w światłach i cieniach.

Naszym celem jest poprawianie jakości i wyciąganie szczegółów
z zdjęć, w których te szczegóły zostały ukryte i utracone w wyniku
niedoświetlenia i prześwietlenia.

\subsection{Wykorzystywane materiały}
Badania będziemy przeprowadzać z wykorzystaniem zdjęć zrobionych
zarówno przy użyciu aparatu analogowego, jak i cyfrowego. Problemy,
które dotykają fotografii analogowej łatwo powtórzyć na aparacie cyfrowym
ustawiając korektę ekspozycji.
Celowo zrobimy zdjęcia jednego ujęcia o różnym poziomie naświetlenia tak,
abyśmy mogli na nich testować algorytm wiedząc, jakich wyników powinniśmy
się spodziewać. Dodatkowo zależy nam na pozyskaniu informacji w jaki sposób zdjęcia doświetlone niepoprawnie odstają od poprawnej ekspozycji.
Zależy nam, aby zdjęcia były o różnorodnej tematyce -- od zdjęć
natury przez portrety po zdjęcia architektury tak, aby mieć pewność, że nasza
metoda ma szerokie zastosowanie w życiu codziennym.


\subsection{Dobór zdjęć}
Będziemy pracować na wysokiej jakości cyfrowych skanach filmu zdjęć
analogowych, a także na typowych dla nas cyfrowych zdjęciach.
Posłużymy się archiwalnymi zdjęciami znalezionymi w
rodzinnych albumach, naszych własnych kolekcjach i wykonanymi
celowo na potrzeby tego projektu.

W tym celu kilkoro członków naszego zespołu chwyciło
za aparaty, i ruszyło fotografować świat!

\newpage
\section{Zdjęcia!}
Kilka przykładowych zdjęć spośród zebranych przez nas, które obrazują problem
utraty szczegółów zdjęć analogowych w przypadku ich niedoświetlenia. \newline

Dalej znajdują się także porównania kilku wybranych zdjęć wykonanych
aparatem cyfrowym przy różnych ustawieniach ekspozycji. Dla zdjęć
cyfrowych najwięcej detali traconych jest w prześwietlonych punktach. \vfill











\begin{figure}[H]
    \centering
    \includegraphics[angle=90, width=\linewidth, keepaspectratio]{Photos1/photos/analog6.jpg}
    \caption{Zdjęcie niedoświetlone.}
\end{figure}
\begin{figure}[H]
    \centering
    \includegraphics[angle=90, width=\linewidth, keepaspectratio]{Photos1/photos/analog7.jpg}
    \caption{Zdjęcie doświetlone.}
\end{figure}



\begin{figure}[H]
    \centering
    \includegraphics[angle=90, width=\linewidth, keepaspectratio]{Photos1/photos/analog11.jpg}
    \caption{Zdjęcie niedoświetlone.}
\end{figure}
\begin{figure}[H]
    \centering
    \includegraphics[angle=90, width=\linewidth, keepaspectratio]{Photos1/photos/analog14.jpg}
    \caption{Zdjęcie doświetlone.}
\end{figure}



\begin{figure}[H]
    \centering
    \includegraphics[angle=90, width=\linewidth, keepaspectratio]{Photos1/photos/analog23.jpg}
    \caption{Zdjęcie niedoświetlone.}
\end{figure}
\begin{figure}[H]
    \centering
    \includegraphics[angle=90, width=\linewidth, keepaspectratio]{Photos1/photos/analog22.jpg}
    \caption{Zdjęcie doświetlone.}
\end{figure}


\begin{figure}[H]
    \centering
    \includegraphics[angle=90, width=\linewidth, keepaspectratio]{Photos1/photos/analog35.jpg}
    \caption{Zdjęcie niedoświetlone.}
\end{figure}
\begin{figure}[H]
    \centering
    \includegraphics[angle=90, width=\linewidth, keepaspectratio]{Photos1/photos/analog36.jpg}
    \caption{Zdjęcie doświetlone.}
\end{figure}


\newpage
A także w przypadku fotografii cyfrowej:

\begin{figure}[H]
    \centering
    \includegraphics[width=\linewidth, keepaspectratio]{Photos1/photos/balony3.jpg}
    \caption{Zdjęcie niedoświetlone}
\end{figure}
\begin{figure}[H]
    \centering
    \includegraphics[width=\linewidth, keepaspectratio]{Photos1/photos/balony2.jpg}
    \caption{Zdjęcie doświetlone}
\end{figure}
\begin{figure}[H]
    \centering
    \includegraphics[width=\linewidth, keepaspectratio]{Photos1/photos/balony1.jpg}
    \caption{Zdjęcie prześwietlone}
\end{figure}


\begin{figure}[H]
    \centering
    \includegraphics[width=\linewidth, keepaspectratio]{Photos1/photos/kot3.jpg}
    \caption{Zdjęcie niedoświetlone}
\end{figure}
\begin{figure}[H]
    \centering
    \includegraphics[width=\linewidth, keepaspectratio]{Photos1/photos/kot2.jpg}
    \caption{Zdjęcie doświetlone}
\end{figure}
\begin{figure}[H]
    \centering
    \includegraphics[width=\linewidth, keepaspectratio]{Photos1/photos/kot1.jpg}
    \caption{Zdjęcie prześwietlone}
\end{figure}

\begin{figure}[H]
    \centering
    \includegraphics[width=\linewidth, keepaspectratio]{Photos1/photos/miod3.jpg}
    \caption{Zdjęcie niedoświetlone}
\end{figure}
\begin{figure}[H]
    \centering
    \includegraphics[width=\linewidth, keepaspectratio]{Photos1/photos/miod1.jpg}
    \caption{Zdjęcie doświetlone}
\end{figure}
\begin{figure}[H]
    \centering
    \includegraphics[width=\linewidth, keepaspectratio]{Photos1/photos/miod2.jpg}
    \caption{Zdjęcie prześwietlone}
\end{figure}


Spora (i ciągle rosnąca) baza zdjęć analogowych i cyfrowych dogłębnie ilustrujących
sedno problemu z którym się zmagamy znajduje się tutaj:
\url{https://drive.google.com/drive/folders/1SFu_A46nBXhL19diXnxV-iyEyH0fnhFA}








\section{Wstępna analiza}
Jakkolwiek często `na oko' względnie łatwo porównując dwa zdjęcia wskazać
to, któremu brakuje szczegółów lub zostało niedoświetlone, to warto się
zastanowić co tak właściwie znaczy to, że zdjęcie jest niedoświetlone. \newline

W celu analizy i próby zdefiniowania zdjęcia o skrajnej jasności
utworzyliśmy kilka Matlabowych skryptów, które analizują wybrane
parametry zdjęć. \newline

\textit{(Uwaga techniczna) W samym raporcie posługujemy się przykładem
    kilku ujęć jednego zdjęcia aby zademonstrować nasze działania, a także
    aby zachować czytelność. Całość dostępna jest tutaj: \url{https://drive.google.com/drive/folders/1SFu_A46nBXhL19diXnxV-iyEyH0fnhFA}.}


\newpage
\subsection{Intensywność}
W przypadku analizy zdjęcia intensywność odnosi się do jasności piksela i
opisuje ogólną ilość światła w obrazie. Jest to miara luminancji, czyli składowej
jasności obrazu, niezależna od koloru. Utworzyliśmy histogramy pokazujące rozłożenie
jasności pikseli na obrazach. Na podstawie tych wykresów można wysunąć wnioski na
temat tego czy zdjęcie jest dobrze oświetlone, niedoświetlone czy prześwietlone.
Zauważyliśmy, że zebrane przez nas zdjęcia analogowe są znacznie ciemniejsze od
tych cyfrowych. Histogramy zdjęć robionych w tych samych warunkach, ale z inną
ekspozycją znacząco różnią się między sobą, jednak ta różnica jest najbardziej
widoczna w przypadku zdjęć cyfrowych.


\subsubsection{Kod}
\begin{verbatim}
    files = dir("photos\*.jpg");
    
    for i = 1:length(files)
    
    clear count g G im k light max n s x y;
    im = imread(strcat("photos\", files(i).name));
    g = rgb2gray(im);
    G = g(:);
    s = length(G);
    figure(1);
    set(gcf, 'Units', 'Normalized', 'OuterPosition', [0 0 1 1]); %wielkość okna
    
    
    subplot(1,3,[1,2]); 
    histogram(G,'FaceColor', '#ffffff');
    
    [count, n] = histcounts( G, 255 );
    max = max(count);
    max = max*1.2;
    
    xlim([0 255]);
    ylim([0,max]);
    
    grid on;
    title('Histogram jasności pikseli','FontSize', 30);
    xlabel('Wartość jasności','FontSize',20);
    ylabel('Ilość pikseli','FontSize',20);
    
    x = [0 0 0 0];
    y = [0 0 max max];
    light = 0.66;
    for k =1:1:5
        hold on;
        x = x + [0 51 51 0];
        patch(x,y,'k','FaceAlpha',light);
        x = x + [51 0 0 51];
        light = light *0.66;
    end
    
    subplot(1,3,3);
    imshow(im);
    
    exportgraphics(gcf, strcat("intensity\", 
                        \\files(i).name(1:length(files(i).name)-4),"_intensity.jpg"))
    close;
    end
    \end{verbatim}


\subsubsection{Wyniki}



\begin{figure}[H]
    \centering
    \includegraphics[width=\textwidth]{Photos1/intensity/jagier1_intensity.jpg}
\end{figure}
\begin{figure}[H]
    \centering
    \includegraphics[width=\textwidth]{Photos1/intensity/jagier2_intensity.jpg}
\end{figure}
\begin{figure}[H]
    \centering
    \includegraphics[width=\textwidth]{Photos1/intensity/jagier3_intensity.jpg}
\end{figure}




\newpage
\subsection{Barwa, odcień}
Hue -- z angielskiego coś pomiędzy barwą a odcieniem w fotografii odnosi się
do podstawowego koloru światła, czyli pozycji danego koloru w spektrum barw
widzialnych. Wyrażany jest w stopniach od $0$ do $360^o$ na kole
barw. Natomiast zmiana hue tylko przesuwa kolor, ale nie zmienia jego jasności
ani nasycenia. Nie badaliśmy wartości hue dla zebranych przez nas zdjęć analogowych,
ponieważ są czarno-białe. Analiza hue dla zdjęć cyfrowych pokazała nieznaczne różnice przy różnej ekspozycji.
\subsubsection{Kod}


\begin{verbatim}
    
    files = dir("photos\*.jpg");
    
    for i = 1:length(files)
    
    clear count g G im max n;
    im = imread(strcat("photos\", files(i).name));
    g = rgb2hsv(im);
    g = g(:,:,1);
    g = g*255;
    G = g(:);
    figure(1);
    set(gcf, 'Units', 'Normalized', 'OuterPosition', [0 0 1 1]); %wielkość okna
    subplot(1,3,[1,2]); 
    histogram(G,'FaceColor', '#ffffff');
    
    [count, n] = histcounts( G, 255 );
    max = max(count);
    max = max*1.2;
    
    xlim([0 255]);
    ylim([0,max]);
    
    grid on;
    title('Histogram odcienia pikseli','FontSize', 30);
    xlabel('Wartość odcienia','FontSize',20);
    ylabel('Ilośc pikseli','FontSize',20);
    
    subplot(1,3,3);
    imshow(im);
    
    exportgraphics(gcf, strcat("hue\", 
                        \\files(i).name(1:length(files(i).name)-4),"_hue.jpg"))
    close;
    
    end
    
  
    
    
    
    
    
    \end{verbatim}
\newpage


\subsubsection{Wyniki}

\begin{figure}[H]
    \centering
    \includegraphics[width=\textwidth]{Photos1/hue/jagier1_hue.jpg}

\end{figure}
\begin{figure}[H]
    \centering
    \includegraphics[width=\textwidth]{Photos1/hue/jagier2_hue.jpg}

\end{figure}
\begin{figure}[H]
    \centering
    \includegraphics[width=\textwidth]{Photos1/hue/jagier3_hue.jpg}

\end{figure}



\newpage
\subsection{Saturacja}
Saturacja w fotografii to stopień intensywności kolorów na zdjęciu. Określa,
jak bardzo barwy są nasycone -- od wyblakłych i niemal czarno-białych (niska
saturacja) do bardzo żywych i intensywnych (wysoka saturacja). Tak samo jak z
hue nie badaliśmy saturacji dla zdjęć czarno-białych. Można zauważyć znaczne
różnice w saturacji zdjęć cyfrowych w zależności od stopnia naświetlenia.

\subsubsection{Kod}
\begin{verbatim}
files = dir("photos\*.jpg");

for i = 1:length(files)

clear count g G im max n;
im = imread(strcat("photos\", files(i).name));
g = rgb2hsv(im);
g = g(:,:,2);
g = g*255;
G = g(:);
figure(1);
set(gcf, 'Units', 'Normalized', 'OuterPosition', [0 0 1 1]); %wielkość okna
subplot(1,3,[1,2]); 
histogram(G,'FaceColor', '#ffffff');

[count, n] = histcounts( G, 255 );
max = max(count);
max = max*1.2;

xlim([0 255]);
ylim([0,max]);

grid on;
title('Histogram nasycenia pikseli','FontSize', 30);
xlabel('Wartość nasycenia','FontSize',20);
ylabel('Ilośc pikseli','FontSize',20);

subplot(1,3,3);
imshow(im);

exportgraphics(gcf, strcat("saturation\", 
            \\files(i).name(1:length(files(i).name)-4),"_saturation.jpg"))
close;

end

\end{verbatim}

\newpage
\subsubsection{Wyniki}


\begin{figure}[H]
    \centering
    \includegraphics[width=\textwidth]{Photos1/saturation/jagier1_saturation.jpg}

\end{figure}
\begin{figure}[H]
    \centering
    \includegraphics[width=\textwidth]{Photos1/saturation/jagier2_saturation.jpg}

\end{figure}
\begin{figure}[H]
    \centering
    \includegraphics[width=\textwidth]{Photos1/saturation/jagier3_saturation.jpg}

\end{figure}

\newpage



\newpage
\subsection{Kontrast}
Kontrast w fotografii to różnica między jasnymi i ciemnymi obszarami zdjęcia.
Określa, jak bardzo elementy obrazu różnią się od siebie pod względem
jasności, koloru lub tonu. Wyższy kontrast sprawia, że zdjęcie wygląda
bardziej dynamicznie, a niski kontrast daje bardziej miękki, wyblakły
efekt.

\subsubsection{Kod}
\begin{verbatim}
    files = dir("photos\*.jpg");
cont = struct('plik', cell(1,length(files)), 'kontrast',
                \\cell(1,length(files)), 'jasnosc', cell(1,length(files)));

for i = 1:length(files)

clear g G l im j k m n R;
im = imread(strcat("photos\", files(i).name)); 
g = rgb2gray(im);
g = single(g);
g=g/255;
G = g(:);
I = mean(G);
[n, m] = size(g);
R =0.0;

for k=1:1:n
    for j = 1:1:m
        R = R + (g(k,j) - I)^2;
    end
end

R = R/(m*n);

R = sqrt(R);

%disp(["Kontrast zdjęcia" R]);
%disp(["Średnia jasność zdjęcia" I]);

cont(i).plik = files(i).name;
cont(i).kontrast = R;
cont(i).jasnosc = I;

end

writetable(struct2table(cont), 'contrast.csv')

\end{verbatim}
\subsubsection{Wyniki}
Co w przypadku naszego zdjęcia dało wyniki następujące:

\begin{table}[h]
    \centering
    \begin{tabular}{|c|c|c|}
        \hline
        Zdjęcie                      & Kontrast  & Jasność    \\ \hline
        Zdjęcie stołu niedoświetlone & 0.1856783 & 0.2637069  \\ \hline
        Zdjęcie stołu idealne        & 0.2608515 & 0.39802570 \\ \hline
        Zdjęcie stołu prześwietlone  & 0.3019148 & 0.54119000 \\ \hline
    \end{tabular}
    \caption{Wyniki dla używanego w raporcie przykładowego zdjęcia stołu}
\end{table}

\newpage
\subsection{Wnioski i plany na przyszłość}
Analiza danych zebranych w powyższych częściach okazała się trudnym zadaniem,
dlatego wrócimy do niej w kolejnym raporcie. Zamieszczamy jednak poniższą tablę,
której poprawności statystycznej dla ogółu danych nie jesteśmy pewni. Na podstawie
danych jesteśmy wciąż w stanie postawić hipotezę, że: im większe doświetlenie tym
wyższy kontrast i jasność. nirzsze


\begin{table}[h]
    \centering
    \begin{tabular}{|c|c|c|c|c|c|}
        \hline
        EV       & -1    & $+\delta$ & 0     & $+\delta$ & +1    \\ \hline
        Kontrast & 0.119 & +37 \%    & 0.163 & +26 \%    & 0.206 \\ \hline
        Jasność  & 0.231 & +27 \%    & 0.294 & +20 \%    & 0.352 \\ \hline
    \end{tabular}
    \caption{Średnie jasności i kontrastu od korekty naświetlenia}

\end{table}




\section{Wykorzystywane narzędzia}
W tej części naszego projektu korzystaliśmy z następujących narzędzi:
\begin{itemize}
    \item Programu Matlab -- do analizy zdjęć;
    \item Programu LibreOffice Calc -- do analizy wyników ankiety;
    \item $\LaTeXe{}$ -- do przygotowania raportu;
    \item Google Drive -- do udostępniania plików;
    \item 7zip -- do kompresji zdjęć;
    \item Aparatów:
          \begin{itemize}
              \item Canon EOS 300 z obiektywem Tamron 28-105mm 1:4-5.6 i kliszą Fomapan 400
              \item Fujifilm FinePix L55 Digital Camera -- Black (12MP, 3x Optical Zoom)
          \end{itemize}
\end{itemize}


\section{Podział obowiązków}
Po wyborze celu projektu wszyscy zajęliśmy się zdobywaniem wiedzy na
temat problemów fotografii analogowej, a także możliwych poprawy jakości zdjęć.\newline


Posiadając wstępną wiedzę na temat materii projektu organicznie wstępnie
podzieliliśmy się zajęciami zgodnie z naszymi zainteresowaniami: \newline
\begin{itemize}
    \item pozyskiwanie materiałów testowych -- Aleksandra Wójcik, Bartosz Wójcik;
    \item opracowanie skryptów do analizy zdjęć i zbieranie informacji do algorytmu -- Katarzyna Szwed, Karol Sęk, Michał Juszkiewicz;
    \item opracowywanie raportu -- Patrycja Szałajko, Natalia Szymańska, Filip Sajko.
\end{itemize}

\newpage































































































\part{}
\thispagestyle{empty}

\begin{figure}[h]
    \centering
    \includegraphics[width=1\textwidth]{wspolne_dla_wszystkich/logo_uczelni.png}
\end{figure}


\begin{center}
    {\LARGE \textbf{Poprawa jakości skanów zdjęć wykonanych techniką analogową
        }} \\[0.3cm]
    {\large \textbf{Raport II}} \\[0.2cm]
    \textit{projekt realizowany pod opieką prof. dr hab. inż. Artura Przelaskowskiego}

\end{center}

\begin{figure}[h]
    \centering
    \includegraphics[width=1\textwidth]{wspolne_dla_wszystkich/logo_projektu.png}
\end{figure}

\vfill
\begin{abstract}
    Raport 2 projektu poprawy jakości cyfrowych skanów zdjęć wykonanych techniką analogową przez grupę nr 9 (wtorkową z godziny 18)
    w składzie:  Bartosz Wójcik, Katarzyna Szwed, Natalia Szymańska,
    Patrycja Szałajko, Aleksandra Wójcik, Karol Sęk, Michał Juszkiewicz, Filip Sajko.

    W tym raporcie zredefiniujemy cel naszego projektu i opiszemy problem z którym się mierzymy.
    Przedstawimy ponadto wstępną wersję naszego programu i zademonstrujemy jego skuteczność.
\end{abstract}


\newpage

% ADD KOREKTA
\section{Korekta do raportu 1}
\begin{itemize}
    \item Do skanowania zdjęć w pierwszym etapie projektu użyty był skaner minilab Noritsu HS-1800.
    \item Zdjęcia były robione przy różnych ustawieniach Exposure Value, za zdjęcia niedoświetlone uznaliśmy te robione przy EV-1, a za prześwietlone przy EV+1.
    \item Również doprecyzowaliśmy tytuł projektu.
\end{itemize}


\section{Cel projektu}
W związku ze słusznymi uwagami i wskazówkami, podjęliśmy decyzję o ukonkretyzowaniu celu naszego projektu.
Skupimy się przede wszystkim na poprawianiu defektów cyfrowych skanów zdjęć analogowych.
Na ten moment pracowaliśmy nad usuwaniem artefaktów powstałych w procesie wywoływania zdjęć i naprawianiem kontrastu zdjęć.
Naszym celem jest zarówno naprawienie jakości starych zdjęć rodzinnych,
jak i tych robionych przez współczesnych amatorów fotografii analogowej.

\section{Zdjęcia, zdjęcia!}
Profilowym zdjęciem dla nas jest portret -- tak indywidualny jak i grupowy.
Jest to typ zdjęć najbardziej popularny w rodzinnych albumach -- mnogość w nich zdjęć z ważnych
dla danej familli wydarzeń: chrztów, wesel czy pogrzebów... Służą one utrwaleniu wspomnień oraz pamięci
po krewnych i bliskich, którzy już odeszli... A więc noszących dużą wartość emocjonalną dla ich posiadacza.

Nasza koleżanka Ola odnalazła stary album rodzinny i zeskanowała znajdujące się w nim zdjęcia.
Docelowo zajmiemy się naprawianiem skanów właśnie tych zdjęć.
Często pojawiającymi się problemami wśród tych zdjęć są niedoświetlenie oraz zagięcia.

\begin{figure}[H]
    \centering
    \includegraphics[width=\linewidth, keepaspectratio]{Photos2/STARE/Scan2025-04-14_101906.png}
    \caption{Przykładowe zdjęcie z albumu, który można znaleźć \href{https://drive.google.com/drive/folders/1FME2DGxQ3jP6B-MKmGzHXzVHgdVP4-Fq}{tutaj!} }
\end{figure}

\newpage
\section{Problemy}
Wykonywanie, a następnie `ucyfrowienie' zdjęcia w technice analogowej wiąże się z różnymi trudnościami,
które mogą znacząco obniżyć jakość zdjęcia -- a z tym satysfakcje jego posiadacza. Głównymi problemami,
którym będziemy przeciwdziałać, będą niedoświetlenie zdjęcia i zanieczyszczenia powietrza osadzające
się na oryginalnym zdjęciu i skanerze podczas procesu zmiany informacji z analogowej na cyfrową.

\subsection{Niedoświetlenie} % tak wiem bartek bedziesz krzyczał. ale poezja to poezja, e viva latre
Niedoświetlenie jest problemem trudnym -- zwłaszcza dla fanów-amatorów techniki analogowej.
Zasadnicza większość klasycznych aparatów nie posiada zaawansowanej mechaniki automatycznie wybierającej
odpowiednie ustawienia aparatu, a brak możliwości podglądu tego, jak dane zdjęcie wyszło, często doprowadza
do sytuacji, gdzie po wielu dniach okazuje się, że na zdjęciu chwili, którą fotograf chciał uchwycić i utrwalić,
niewiele widać, bo przez złe ustawienia większość szczegółów jest niewidoczna. Jest to problem, który szeroko
wraz z przykładami i analizą numeryczną opisywaliśmy w raporcie pierwszym.

Dla przykładu przypomnijmy:
\newpage
\begin{figure}[H]
    \centering
    \includegraphics[angle=90, width=\linewidth, keepaspectratio]{Photos2/doswietlone_i_nie/analog6.jpg}
    \caption{Zdjęcie niedoświetlone.}
\end{figure}
\begin{figure}[H]
    \centering
    \includegraphics[angle=90, width=\linewidth, keepaspectratio]{Photos2/doswietlone_i_nie/analog7.jpg}
    \caption{...i to w punkt.}
\end{figure}


% REWORD THIS 

\newpage
\subsection{Artefakty}
Artefakty na zdjęciach analogowych powstają ze względu na osadzanie się kurzu i innych zanieczyszczeń na wywołanej kliszy,
a także przez jej zarysowanie.

Przez to na negatywie powstają ciemne plamki, które w momencie powstawania odbitki z kolei zamieniają się w jasne plamy.

Takie zjawisko można zaobserwować na poniższych zdjęciach:

\begin{figure}[H]
    \centering
    \includegraphics[width=\linewidth, keepaspectratio]{Photos2/przed/GRAYgpt3.png}
    \caption{W skrajnych wypadkach może wyglądać to nawet tak.}
\end{figure}

\begin{figure}[H]
    \centering
    \includegraphics[width=\linewidth, keepaspectratio]{Photos2/przed/new1.jpeg}
    \caption{Choć bardziej częstym jest ten przypadek.}
\end{figure}

\begin{figure}[H]
    \centering
    \includegraphics[width=\linewidth, keepaspectratio]{Photos2/przed/new2.jpeg}
    \caption{A także taki.}
\end{figure}

\newpage
\section{Program i jego działanie}
Zbrojni w wiedzę co chcemy osiągnąć i zapas zebranych skanów zdjęć do testów wzięliśmy się do pracy
nad programem. Stworzyliśmy zaawansowany program, który inteligentnie przeszukuje cały obszar zdjęcia, zaznacza artefakty,
a w następnej fazie działania usuwa je. Warto dodać, że algorytm przekształca zdjęcie z formatu RGB do formatu HSV (Hue, Saturation, Value) i działa tylko na Value,
która definiuje jasność piksela w skali 0-255.

Przechodząc do szczegółów, działanie programu można opisać za pomocą kilku kolejnych faz działania, na przykładzie poniższego zdjęcia:

\begin{figure}[H]
    \centering
    \includegraphics[width=\linewidth, keepaspectratio]{Photos2/przed/gpt1.png}
    \caption{Przykładowe zdjęcie -- to za jego pomocą opiszemy działanie programu.}
\end{figure}


\subsection{Tworzenie maski}
\subsubsection{Rozmycie gaussowskie}
Pierwszym krokiem generowania maski jest stworzenie rozmycia gaussowskiego zdjęcia.

Wygładzanie gaussowskie jest efektem rozmywania obrazu za pomocą funkcji Gaussa,
która jest szeroko wykorzystywana w grafice komputerowej w celu uzyskania gładkiego
wygładzenia obrazu i wyciszenia szumu informacyjnego.

Za pomocą funkcji danej wzorem dla każdego piksela:
\begin{equation}
    G(x, y) = \frac{1}{2 \pi \sigma^2}e^{- \frac{x^2+y^2}{2\sigma^2}}
\end{equation}
gdzie $x$, $y$ to współrzędne danego piksel'a, a $\sigma$ oznacza odchylenie standardowe.\footnote{Źródło: \url{https://en.wikipedia.org/wiki/Gaussian_blur}}

\begin{figure}[H]
    \centering
    \includegraphics[width=\linewidth, keepaspectratio]{Photos2/gauss_blurr/gauss_blurr_gpt1.png}
    \caption{Zdjęcie po wykonaniu rozmycia gaussowskiego.}
\end{figure}

\subsubsection{Wykonanie różnicy}
Następnie odejmujemy od oryginału zdjęcia otrzymane w poprzednim etapie rozmycie.
Pozwala to na wykrycie najbardziej kontrastowych elementów zdjęcia.

\begin{figure}[H]
    \centering
    \includegraphics[width=\linewidth, keepaspectratio]{Photos2/difference/difference_gpt1.png}
    \caption{Zdjęcie po wykonaniu różnicy.}
\end{figure}

\subsubsection{Usunięcie ciemnych pikseli}
Zostawiamy tylko jasne piksele (według standardowych ustawień jest
to jasność powyżej 60) i nadajemy im maksymalną wartość 255.
Pozostałym pikselom ustawiamy jasność na 0.

\newpage
\subsubsection{Korekcja gamma}

Korekcja gamma jest techniką stosowaną w grafice komputerowej, której celem jest dostosowanie
jasności obrazu do ludzkiego nielinowego postrzegania światła. Korekcja gamma kompensuje ten
nieliniowy sposób widzenia, pozwalając na efektywniejsze wykorzystanie dostępnych poziomów jasności.

Korekcja gamma dokonuje transformacja jasności w następujący sposób:
\begin{equation}
    L'(x,y) = L(x,y)^{\gamma}
\end{equation}
gdzie:
\begin{itemize}
    \item $\gamma$ to współczynnik gamma,
    \item $L(x,y)$ to wartości jasności pikseli.
\end{itemize}

Wiedząc to, w tym etapie bierzemy ponownie oryginalne zdjęcie, wykonujemy
na nim korekcję gamma, a następnie na tym zdjęciu wykonujemy etapy 1-3.

Działanie to pozwala uwzględnić jak najwięcej artefaktów
-- znajdujących się także na jasnym tle.

\begin{figure}[H]
    \centering
    \includegraphics[width=\linewidth, keepaspectratio]{Photos2/gamma_corection/gamma_corection_gpt1.png}
    \caption{Po wykonaniu korekcji gamma i etapów 1-3.}
\end{figure}

\newpage
\subsection{Gotowa maska}
Po tym wszystkim otrzymujemy dwie podmaski (jedna robiona na oryginalnym
zdjęciu, a druga na jaśniejszym -- rozświetloną modulacją gamma).
Końcową maskę otrzymujemy biorąc wszystkie znalezione piksele z obu podmasek.

\begin{figure}[H]
    \centering
    \includegraphics[width=\linewidth, keepaspectratio]{Photos2/masks/final_mask_gpt1.png}
    \caption{Maska wykonana z oganianego zdjęcia}
\end{figure}
\begin{figure}[H]
    \centering
    \includegraphics[width=\linewidth, keepaspectratio]{Photos2/masks/second(gc)_mask_gpt1.png}
    \caption{Maska wykonana z zdjęcia rozświetlonego korekcją gamma}
\end{figure}


\newpage
\subsection{Działanie właściwe}
Mając wygenerowaną maskę właściwą, przechodzimy po wyznaczonych
przez nią pikselach na oryginalnym zdjęciu. Dla każdego piksela w
zaznaczonego w masce wyliczamy nową wartość jasności -- średnią z
jasności wszystkich (nie zaliczają się do tego pikseli wyznaczone
wcześniej przez maskę.) pikseli na odległości nie więcej 15 od aktualnie analizowanego.

Wynik finalny:
\begin{figure}[H]
    \centering
    \includegraphics[width=\linewidth, keepaspectratio]{Photos2/przed/gpt1.png}
    \includegraphics[width=\linewidth, keepaspectratio]{Photos2/po/gpt1.png}
    \caption{Przykładowe zdjęcie -- porównanie efektu przed i po.}

\end{figure}

\newpage
\section{Uwagi co do działania programu}
Przez to, że algorytm działa lokalnie (tylko w punktach wyznaczonych
przez maskę) nie wpływa on na ogólny wygląd zdjęcia.
Z tego też powodu możemy wykonywać program kilkukrotnie na zdjęciu
-- uzyskując lepsze efekty.

Końcowy algorytm składa się z pięciu iteracji opisanego wyżej algorytmu,
znacząco zwiększając szansę na usunięcie zanieczyszczeń.

Ponadto można uzyskać dodatkowe informacje o działaniu programu
i jakości zdjęcia -- za przykład funkcja: .countNoise()
zlicza ilość wyznaczonych przez maskę pikseli -- które są uznane za zanieczyszczenia.


\section{Program do zmiany kontrastu zdjęcia}
Napisaliśmy też osobny program zmieniający kontrast obrazu.
Użyliśmy języka Matlab ze względu na to, że umożliwia on łatwiejsze porównanie wyników ze względu na parametr gamma.
Algorytm korzysta z korekcji gamma opisanej wcześniej w punkcie 4.1.4.

\subsection{Kod}
\begin{verbatim}
g = imread("test_2.png");
figure(1);
tiledlayout(3,3)
nexttile
imshow(g);
title("zdjęcie niezmienione")
gamma = 0.25;
for k=1:9
    if gamma ~= 1
        gout = double(g)/255;
        gout = gout.^gamma;
        gout = gout*255;
        gout = uint8(gout);
        nexttile
        imshow(gout);
        title(['gamma = ' num2str(gamma)]);
    end
    gamma = gamma + 0.25;
end
\end{verbatim}

\newpage
\subsection{Działanie programu ze względu na parametr gamma}
\begin{figure}[H]
    \centering
    \includegraphics[width=\textwidth, keepaspectratio]{Photos2/porownanie.png}
    \caption{Działanie programu dla różnych parametrów $\gamma$}
\end{figure}


\section{Dalsze kroki}
Na chwilę obecną nasze rozwiązanie jest programem terminalowym,
działającym na systemie nie starszym niż Windows 10 -- choć istnieje
plan przeniesienia go także na inne popularne systemy operacyjne.

Tak samo pracujemy obecnie nad stworzeniem bardziej przystępnego interfejsu okienkowego, a przy tym
planujemy integrację poszczególnych elementów programu. Zamierzamy również między innymi dodać funkcjonalność usuwającą widoczne zagięcia.

Program dostępny jest na licencji \textit{open source} i jego kod źródłowy można znaleźć na GitHubie
pod adresem:
\begin{center}
    \url{https://github.com/ssk12o/PTI-Foto-Projekt}.
\end{center}




\newpage
\section{Wykorzystywane narzędzia}
W tej części naszego projektu korzystaliśmy z następujących narzędzi:
\begin{itemize}
    \item Programu i języka Matlab -- do analizy zdjęć i kontrastu;
    \item Języka C++ -- do napisania programu usuwającego artefakty;
    \item Programu VS Code -- do tworzenia, edycji i dokumentacji kodu programu i raportów;
    \item Programu LibreOffice Calc -- do analizy części danych numerycznych;
    \item $\LaTeXe{}$ -- do przygotowania raportu;
    \item Strony Github i programu Git -- do udostępniania, dystrybucji i pracy nad kodem;
    \item 7zip -- do kompresji zdjęć;
    \item Google Drive -- do udostępniania plików;
    \item Skanera minilab Noritsu HS-1800 -- do wykonywania wysokiej jakości cyfrowych skanów zdjęć wykonanych techniką analogową;
    \item Aparatów:
          \begin{itemize}
              \item Canon EOS 300 z obiektywem Tamron 28-105mm 1:4-5.6 i kliszą Fomapan 400
              \item Fujifilm FinePix L55 Digital Camera -- Black (12MP, 3x Optical Zoom)
          \end{itemize}
\end{itemize}


\section{Podział obowiązków}
Na tym etapie projektu podzieliśmy się pracą, obowiązkami i zadaniami w następujący sposób:
\begin{itemize}
    \item Bartosz Wójcik -- wykonywanie, skanowanie i analiza zdjęć; research.
    \item Katarzyna Szwed -- korekta raportu; tworzenie, analizowanie i pisanie algorytmu.
    \item Natalia Szymańska -- pisanie raportu.
    \item Patrycja Szałajko -- zarządzanie pracą zespołu, kontakt.
    \item Aleksandra Wójcik -- skanowanie zdjęć rodzinnych w celu polepszenia ich jakości w końcowych etapach projektu.
    \item Karol Sęk -- tworzenie, analizowanie i pisanie algorytmu.
    \item Michał Juszkiewicz -- tworzenie, analizowanie i pisanie algorytmu.
    \item Filip Sajko -- pisanie raportu, implementacja w \LaTeX{}.
\end{itemize}


%                       Wersja alternatywna podziału obowiązków
% \section{Podział obowiązków}
% Na tym etapie projektu podzieliśmy się pracą, obowiązkami i zadaniami w następujący sposób:
% 
% \begin{table}[h!]
%     \centering
%     \renewcommand{\arraystretch}{1.3}
%     \begin{tabular}{|p{3cm}|p{7.5cm}|} \hline
%         \textbf{Imię i nazwisko} & \textbf{Zakres obowiązków}                                                               \\ \hline \hline
%         Bartosz Wójcik           & Wykonywanie, skanowanie i analiza zdjęć; opieka merytoryczna.                            \\ \hline
%         Katarzyna Szwed          & Tworzenie, analizowanie i pisanie algorytmu; korekta raportu.                            \\ \hline
%         Natalia Szymańska        & Pisanie raportu.                                                                         \\ \hline
%         Patrycja Szałajko        & Zarządzanie pracą zespołu, kontakt z mediami.                                            \\ \hline
%         Aleksandra Wójcik        & Skanowanie zdjęć rodzinnych w celu polepszenia ich jakości w końcowych etapach projektu. \\ \hline
%         Karol Sęk                & Tworzenie, analizowanie i pisanie algorytmu.                                             \\ \hline
%         Michał Juszkiewicz       & Tworzenie, analizowanie i pisanie algorytmu.                                             \\ \hline
%         Filip Sajko              & Pisanie raportu, implementacja w \LaTeX{}.                                               \\ \hline
%     \end{tabular}
%     \caption{Podział obowiązków w zespole projektowym.}
% \end{table}


















































\newpage

\part{}
\thispagestyle{empty}

\begin{figure}[h]
    \centering
    \includegraphics[width=1\textwidth]{wspolne_dla_wszystkich/logo_uczelni.png}
\end{figure}


\begin{center}
    {\LARGE \textbf{Poprawa jakości skanów zdjęć wykonanych techniką analogową
        }} \\[0.3cm]
    {\large \textbf{Raport III}} \\[0.2cm]
    \textit{projekt realizowany pod opieką prof. dr hab. inż. Artura Przelaskowskiego}

\end{center}

\begin{figure}[h]
    \centering
    \includegraphics[width=1\textwidth]{wspolne_dla_wszystkich/logo_projektu.png}
\end{figure}

\vfill
\begin{abstract}
    Raport 3 projektu poprawy jakości cyfrowych skanów zdjęć wykonanych techniką analogową przez grupę nr 9 (wtorkową z godziny 18)
    w składzie:  Bartosz Wójcik, Katarzyna Szwed, Natalia Szymańska,
    Patrycja Szałajko, Aleksandra Wójcik, Karol Sęk, Michał Juszkiewicz, Filip Sajko.

    W tym raporcie pokażemy co robiliśmy w kierunku poprawy działania programu i opiszemy te poprawki.
    Zajmiemy się ponadto ekstensywnym testowaniem naszego programu i wyciągnięciem wniosków na temat jego działania i optymalnych ustawień.
\end{abstract}

% \newpage
% \tableofcontents{}

\newpage

\section{Korekta do raportu 2}
\begin{itemize}
    \item Dla zdjęcia dziewczyny z poprzedniego raportu szerokość filtru gaussowskiego $ =  37$ pikseli.
\end{itemize}


\section{Rozszerzanie działalności programu}
\subsection{Słowem wstępu...}
Głównym celem naszego programu jest coś, co można nazwać
ogólnie (konkretnie nasz cel zdefiniowaliśmy w poprzednim raporcie)
poprawą jakości skanów zdjęć analogowych. Trzeba jednak pamiętać,
że najważniejszym ogniwem każdego problemu jest człowiek, którego
problem ten dotyczy -- i któremu chce się pomóc oferując dane rozwiązania. \newline

Tak więc łatwo zauważyć, że kluczowym elementem każdego programu jest dobrze wykonany
interfejs użytkownika. W ostatecznym rozrachunku przecież gorszy
jest przecież program z najlepszym algorytmem, ale złym interfejsem, z którego użytkownik i tak nie skorzysta. \newline


Rozumiejąc tę potrzebę wraz z optymalizacją i rozszerzaniem funkcjonalności
naszego programu, stworzyliśmy interfejs użytkownika i zaczęliśmy % tja... na pewno
tworzyć szczegółową instrukcję obsługi naszego programu.

\subsection{Menu programu}
Obecnie główne menu naszego programu oferuje następujące opcje:

\begin{table}[h]
    \centering
    \begin{tabular}{|c|l|}
        \hline
        Lp. & Opcja                                 \\ \hline
        1   & usuwanie farfocli                     \\ \hline       %
        2   & normalizacja histogramu jasności      \\ \hline       %
        3   & filtracja filtrem bilateralnym        \\ \hline       %
        4   & filtracja filtrem gaussowskim         \\ \hline       %
        5   & usuwanie szumu -- uśrednianie pikseli \\ \hline       %
        6   & wyostrzenie -- maska wyostrzająca     \\ \hline       %
    \end{tabular}
    \caption{Opcje główne naszego programu dostępne dla użytkownika}
\end{table}

Teraz zajmiemy się szczegółowym omówieniem wymienionych w powyższej tabeli opcji. \newpage






\section{Usuwanie farfocli}

\subsection{Słowem wstępu}
Jest to jeden z głównych celów naszego projektu i problem z którym się mierzyliśmy podczas poprzedniego etapu.

Szczegółowo sposób wykrywania, usuwania tego typu artefaktów i normalizowania
zdjęć nimi zanieczyszczonych opisaliśmy w poprzednim raporcie.

W tej części chcielibyśmy zająć się głównie opisaniem efektów naszej pracy.

\subsection{Efekty ogólne i ciekawe przypadki}
W prawie wszystkich testowanych przez nas przypadkach działanie programu w sposób widoczny poprawiało jakość i estetykę zdjęcia.

Nasz algorytm poza usuwaniem artefaktów opisywanych przez nas jako farfocle dobrze radzi sobie z usuwaniem nacieków, można
to zauważyć na poniższych zdjęciach:


\newpage

\begin{figure}[H]
    \centering
    \includegraphics[width=\linewidth, keepaspectratio]{Photos2/przed/gpt1.png}
    \caption{Zdjęcie dziewczyny z wyraźnymi farfoclami.}
\end{figure}
\begin{figure}[H]
    \centering
    \includegraphics[width=\linewidth, keepaspectratio]{Photos2/po/gpt1.png}
    \caption{Zdjęcie poprawione, szerokość filtru gaussowskiego $37$ pikseli.}
\end{figure}

\begin{figure}[H]
    \centering
    \includegraphics[width=\linewidth, keepaspectratio]{Photos3/1 farfocle/pogrzeb_before.png}
    \caption{Zdjęcie zarysowane.}
\end{figure}
\begin{figure}[H]
    \centering
    \includegraphics[width=\linewidth, keepaspectratio]{Photos3/1 farfocle/pogrzeb_after.png}
    \caption{Zdjęcie poprawione, szerokość filtru gaussowskiego $29$ pikseli.}
\end{figure}

\begin{figure}[H]
    \centering
    \includegraphics[width=\linewidth, keepaspectratio]{Photos3/1 farfocle/dziewczyna_before.jpg}
    \caption{Zdjęcie z zaciekami.}
\end{figure}
\begin{figure}[H]
    \centering
    \includegraphics[width=\linewidth, keepaspectratio]{Photos3/1 farfocle/dziewczyna_after.jpg}
    \caption{Poprawione zdjęcie, szerokość filtru gaussowskiego $3$ piksele.}
\end{figure}

\begin{figure}[H]
    \centering
    \includegraphics[width=\linewidth, keepaspectratio]{Photos3/1 farfocle/farfocle4_before.jpg}
    \caption{Zdjęcie z zaciekami.}
\end{figure}
\begin{figure}[H]
    \centering
    \includegraphics[width=\linewidth, keepaspectratio]{Photos3/1 farfocle/farfocle4_after.jpg}
    \caption{Zdjęcie poprawione, szerokość filtru gaussowskiego $3$ piksele.}
\end{figure}




\newpage

\section{Normalizacja histogramu jasności       }
\subsection{Metoda}
Wyrównanie histogramu jasności jest metodą przetwarzania obrazów, polegającą na
poprawianiu kontrastu zdjęcia z wykorzystaniem jego histogramu -- który się normalizuje.
Metoda ta rozciąga wartości pikseli z obrazu na cały przedział jasności.
Dzięki temu, fragmenty zdjęcia o pierwotnie niskim poziomie kontrastu stają się czytelniejsze.


\subsection{Przykłady}
Metoda ta jest bardzo skuteczna w poprawianiu kontrastu zdjęcia -- niestety czasem
kosztem jakości zdjęcia, ponieważ nie rozróżnia ona sygnału od szumu często powoduje ona wzrost wartości szumu w przypadku, kiedy
oryginalne zdjęcie ma mały kontrast. Widać to na poniższym zdjęciu:


\begin{figure}[H]
    \centering
    \includegraphics[width=\linewidth, keepaspectratio]{Photos3/normalizacja histogramu/test2m.PNG}
    \caption{Zdjęcie przed normalizacją wraz z histogramem.}
\end{figure}
\begin{figure}[H]
    \centering
    \includegraphics[width=\linewidth, keepaspectratio]{Photos3/normalizacja histogramu/test2cm.PNG}
    \caption{Zdjęcie po normalizacji (zaszumione) wraz z histogramem.}
\end{figure}

\newpage

\begin{figure}[H]
    \centering
    \includegraphics[width=\linewidth, keepaspectratio]{Photos3/2 normalizacja histogramu jasności/hist/motor_intensity.jpg}
    \caption{Zdjęcie przed normalizacją wraz z histogramem.}
\end{figure}
\begin{figure}[H]
    \centering
    \includegraphics[width=\linewidth, keepaspectratio]{Photos3/2 normalizacja histogramu jasności/hist/motor_after_intensity.jpg}
    \caption{Zdjęcie po normalizacji wraz z histogramem. Szum jest dużo mniejszy niż na poprzednim przykładzie.}
\end{figure}




\vfill
\section{Filtracja filtrem bilateralnym         }
\subsection{Metoda}
Filtr bilateralny jest nieliniowym filtrem wygładzającym, przede wszystkim charakteryzuje się tym,
że zachowuje krawędzie i redukuje szum.

Zastępuje on intensywność każdego piksela średnią ważoną intensywności pobliskich pikseli
wyliczoną poprzez filtr o odpowiednim rozmiarze.
Jeżeli wartość danego piksela jest bliska wartości piksela w centrum filtru, jego waga jest
oparta na rozkładzie Gaussa. W przeciwnym wypadku waga jest bliska 0.
Ta złożona zależność pozwala zachować ostre krawędzie przy jednoczesnym tłumieniu szumu.

Po zastosowaniu algorytmu zauważyliśmy bardzo niewielkie zmiany, zatem kontynuowaliśmy poszukiwanie lepszych rozwiązań.
\newpage


\subsection{Przykłady}

\begin{figure}[H]
    \centering
    \includegraphics[angle=90, width=\linewidth, keepaspectratio]{Photos3/filtr bilateralny/test3.jpg}
    \caption{Zdjęcie pierwotne.}
\end{figure} \newpage
\begin{figure}[H]
    \centering
    \includegraphics[width=\linewidth, keepaspectratio]{Photos3/filtr bilateralny/test3b.jpg}
    \caption{Zdjęcie po zastosowaniu filtru bilateralnego.}
\end{figure}






\section{Filtracja filtrem gaussowskim           }
\subsection{O rozmyciu gaussowskim}
Filtracja filtrem gaussowskim oparta jest na generowaniu maski poprzez tworzenie rozmycia gaussowskiego zdjęcia.

Wygładzanie gaussowskie jest efektem rozmywania obrazu za pomocą funkcji rozkładu Gaussa,
która jest szeroko wykorzystywana w grafice komputerowej w celu uzyskania wygładzenia obrazu i wyciszenia szumu informacyjnego.

Wygładzenie paradoksalnie pomimo utraty ostrości obrazu przyczynia się do poprawy wizualnego odbioru zdjęcia, jednak
tak jak w przypadku filtru bilateralnego zmiany były mało zauważalne.

\subsection{Przykłady}

\begin{figure}[H]
    \centering
    \includegraphics[angle=90, width=\linewidth, keepaspectratio]{Photos3/filtr gaussowski/test3.jpg}
    \caption{Zdjęcie przed.}
\end{figure} \newpage
\begin{figure}[H]
    \centering
    \includegraphics[width=\linewidth, keepaspectratio]{Photos3/filtr gaussowski/test3g.jpg}
    \caption{Zdjęcie po.}
\end{figure}





\vfill
\section{Usuwaniem szumu -- uśrednianie pikseli}
\subsection{Metoda}
Ta metoda polega na uśrednianiu wartości pikseli w oparciu o piksele do nich podobne.

Dane piksele uznaje się za podobne, jeśli średnia jasność ich otoczenia
jest zbliżona. Następnie, wartość danego piksela jest zastępowana
średnią wartością wszystkich pikseli podobnych do niego.


W przeciwieństwie do innych metod usuwania szumu, metoda ta nie jest lokalna,
dla każdego piksela piksele podobne mogą się znajdować gdziekolwiek na zdjęciu.

Algorytm ten usuwa szum nie niszcząc struktury i detali zdjęcia, ze wszystkich trzech
wypróbowanych przez nas metod na usuwanie szumu ta sprawdza się najlepiej.
\newpage


\subsection{Przykłady}
\begin{figure}[H]
    \centering
    \includegraphics[width=\linewidth, keepaspectratio]{Photos3/5 szum/srodmiescie_before_szum.jpg}
    \caption{Zdjęcie przed usuwaniem szumu.}
\end{figure}
\begin{figure}[H]
    \centering
    \includegraphics[width=\linewidth, keepaspectratio]{Photos3/5 szum/srodmiescie_after_szum.jpg}
    \caption{Zdjęcie po usuwaniu szumu.}
\end{figure} \newpage

\begin{figure}[H]
    \centering
    \includegraphics[width=\linewidth, keepaspectratio]{Photos3/5 szum/motor_before_szum.jpg}
    \caption{Zdjęcie przed usuwaniem szumu.}
\end{figure}
\begin{figure}[H]
    \centering
    \includegraphics[width=\linewidth, keepaspectratio]{Photos3/5 szum/motor_after_szum.jpg}
    \caption{Zdjęcie po usuwaniu szumu.}
\end{figure}

\begin{figure}[H]
    \centering
    \includegraphics[width=\linewidth, keepaspectratio]{Photos3/6 wyostrzanie maską wyostrzającą/pogrzeb_ostrosc.png}
    \caption{Zdjęcie przed usuwaniem szumu.}
\end{figure}
\begin{figure}[H]
    \centering
    \includegraphics[width=\linewidth, keepaspectratio]{Photos3/6 wyostrzanie maską wyostrzającą/pogrzeb_odszum.png}
    \caption{Zdjęcie po usuwaniu szumu.}
\end{figure}



\newpage
\section{Wyostrzenie -- maska wyostrzająca      }

\subsection{Metoda}
Wyostrzanie za pomocą maski nieostrości jest techniką wyostrzania obrazu
powszechnie wykorzystywaną w obróbce fotografii w celu poprawy ostrości zdjęć.

Nazwa pochodzi od faktu, że technika ta wykorzystuje rozmyty,
`nieostry' negatyw obrazu do stworzenia maski obrazu oryginalnego.
Maska nieostra jest następnie łączona z oryginalnym pozytywem,
tworząc obraz mniej rozmyty niż pierwotny. Algorytm pomimo wyostrzenia obrazu dodaje
do niego szum, z którym możemy walczyć za pomocą zaimplementowanych przez nas algorytmów odszumiających.

\subsection{Przykłady}
\begin{figure}[H]
    \centering
    \includegraphics[width=\linewidth, keepaspectratio]{Photos3/6 wyostrzanie maską wyostrzającą/dzieci.png}
    \caption{Zdjęcie przed wyostrzeniem maską wyostrzającą.}
\end{figure}
\begin{figure}[H]
    \centering
    \includegraphics[width=\linewidth, keepaspectratio]{Photos3/6 wyostrzanie maską wyostrzającą/dzieci_ostrosc.png}
    \caption{Zdjęcie po wyostrzeniu maską wyostrzającą.}
\end{figure}

\begin{figure}[H]
    \centering
    \includegraphics[angle=-90, width=\linewidth, keepaspectratio]{Photos3/6 wyostrzanie maską wyostrzającą/dziadek.png}
    \caption{Zdjęcie przed wyostrzeniem maską wyostrzającą.}
\end{figure}
\begin{figure}[H]
    \centering
    \includegraphics[angle=-90, width=\linewidth, keepaspectratio]{Photos3/6 wyostrzanie maską wyostrzającą/dziadek_ostrosc.png}
    \caption{Zdjęcie po wyostrzeniu maską wyostrzającą.}
\end{figure}








\newpage
\section{Przykładowe wykorzystanie programu jako całości}
Dzięki naszym badaniom wypracowaliśmy narzędzie o szerokim zastosowaniu w życiu codziennym.
Aby zmaksymalizować jego skuteczność, przeprowadziliśmy dodatkowe badania,
mające na celu ustalenie najefektywniejszej metody użycia. Odkryliśmy, że najlepsze efekty osiągamy,
stosując kolejno narzędzia do usuwania farfocli, poprawiania ostrości oraz odszumiania.

Dzieje się tak dlatego, że najpierw eliminujemy artefakty,
powstałe podczas wywoływania i skanowania zdjęcia.
Kolejny etap -- poprawianie ostrości -- zwiększa jednak poziom szumu,
dlatego końcowe odszumianie cofa niepożądane skutki uboczne poprzedniego kroku.

Trzeba jednak podkreślić, że nasze narzędzie nie jest na tym etapie w pełni zautomatyzowane
-- każde zdjęcie wymaga indywidualnej obróbki, dostosowanej do specyfiki problemu.
Widać to, analizując fotografie użyte w badaniach: w zależności od potrzeb stosowaliśmy różne maski,
które dopasowywaliśmy do konkretnej sytuacji.




\section{Dostępność programu}
Na chwilę obecną nasze rozwiązanie jest zaawansowanym programem terminalowym,
działającym na każdym systemie z dostępną i zainstalowaną biblioteką OpenCV oraz
dowolnym kompilatorem języka C++ zdolnym do kompilacji na tym systemie lub jego pochodnym.

Program dostępny jest na licencji \textit{open source} i jego kod źródłowy można znaleźć na GitHubie
pod adresem:
\begin{center}
    \url{https://github.com/ssk12o/PTI-Foto-Projekt}.
\end{center}
skąd można go łatwo pobrać, skompilować i wykorzystywać dowolnie (choć zgodnie z przeznaczeniem!).




\newpage
\section{Wykorzystywane narzędzia}
W tej części naszego projektu korzystaliśmy z następujących narzędzi:
\begin{itemize}
    \item Programu i języka Matlab -- do analizy zdjęć;
    \item Języka C++ -- do napisania programu usuwającego artefakty;
    \item Programu VS Code -- do tworzenia, edycji i dokumentacji kodu programu i raportów;
    \item Programu LibreOffice Calc -- do analizy części danych numerycznych;
    \item $\LaTeXe{}$ -- do przygotowania raportu;
    \item Strony Github i programu Git -- do udostępniania, dystrybucji i pracy nad kodem;
    \item 7zip -- do kompresji zdjęć;
    \item Google Drive -- do udostępniania części dużych plików;
    \item Skanera minilab Noritsu HS-1800 -- do w dalszej części wykonywania wysokiej jakości cyfrowych skanów zdjęć wykonanych techniką analogową;
    \item Aparatów:
          \begin{itemize}
              \item Canon EOS 300 z obiektywem Tamron 28-105mm 1:4-5.6 i kliszą Fomapan 400
              \item Fujifilm FinePix L55 Digital Camera -- Black (12MP, 3x Optical Zoom)
          \end{itemize}
\end{itemize}


\section{Podział obowiązków}
Na tym etapie projektu podzieliśmy się pracą, obowiązkami i zadaniami w następujący sposób:
\begin{itemize}
    \item Bartosz Wójcik -- wykonywanie, skanowanie i analiza zdjęć; research.
    \item Katarzyna Szwed -- korekta raportu; analiza zdjęć i działania programu.
    \item Natalia Szymańska -- pisanie raportu.
    \item Patrycja Szałajko -- zarządzanie pracą zespołu, kontakt.
    \item Aleksandra Wójcik -- skanowanie zdjęć rodzinnych w celu polepszenia ich jakości w końcowych etapach projektu.
    \item Karol Sęk -- tworzenie, analizowanie i pisanie algorytmu.
    \item Michał Juszkiewicz -- tworzenie, analizowanie i pisanie algorytmu.
    \item Filip Sajko -- pisanie raportu, implementacja w \LaTeXe{}.
\end{itemize}









































































\newpage
\part{}
\thispagestyle{empty}

% \begin{figure}[h]
%     \centering
%     \includegraphics[width=1\textwidth]{wspolne_dla_wszystkich/logo_uczelni.png}
% \end{figure}


\begin{center}
    {\LARGE \textbf{Poprawa jakości skanów zdjęć wykonanych techniką analogową
        }} \\[0.3cm]
    {\large \textbf{Raport IV}} \\[0.2cm]
    \textit{projekt realizowany pod opieką prof. dr hab. inż. Artura Przelaskowskiego}

\end{center}

\begin{figure}[h]
    \centering
    \includegraphics[width=1\textwidth]{wspolne_dla_wszystkich/logo_projektu.png}
\end{figure}

\vfill
\begin{abstract}
    Raport 4 projektu poprawy jakości cyfrowych skanów zdjęć wykonanych techniką analogową przez grupę nr 9 (wtorkową z godziny 18)
    w składzie:  Bartosz Wójcik, Katarzyna Szwed, Natalia Szymańska,
    Patrycja Szałajko, Aleksandra Wójcik, Karol Sęk, Michał Juszkiewicz, Filip Sajko.

    Na tym etapie projektu dodaliśmy tryb automatycznej korekty zdjęcia --
    dodaliśmy presety wartości do poszczególnych filtrów oraz tryb automatycznego doboru filtrów.
    Będziemy również porównywać nasz program z popularnymi narzędziami do obróbki zdjęć - Gimp, PhotoGlory oraz HotpotAI.
    Na podstawie zestawienia wyników otrzymanych za pomocą tych trzech narzędzi ustalimy wady i zalety naszego rozwiązania.
\end{abstract}


\newpage

% \section{Korekta do raportu 3}
% \begin{itemize}
%     \item
%     \item
%     \item
%     \item
% \end{itemize}


\section{Tryb automatyczny (presety)}
Każdy z filtrów dostępnych w naszym programie oblicza domyślne
wartości dla danego zdjęcia na podstawie kontrastu, ostrości i
poziomu zaszumienia zdjęcia.
W programie nadal pozostaje opcja ręcznego ustawiania wartości
dla niestandardowych zdjęć.

Wprowadziliśmy także metodę adaptacyjną, wykorzystuje ona trzy
filtry do poprawy zdjęcia:
\begin{center}
    Normalizacja histogramu jasności $\to$ Filtr bilateralny $\to$ Wyostrzenie zdjęcia
\end{center}

Dokładne działanie pojedynczych filtrów opisaliśmy we wcześniejszych raportach.

Program dobiera odpowiednie wartości dla filtrów (lub który filtr zaaplikować)
na podstawie 3 metryk - odchylenie standardowe jasności dla filtru
korekty histogramu jasności, odchylenie standardowe szumu dla
filtru bilateralnego, wartości akutancji zdjęcia dla filtru
poprawy ostrości. Filtr wyostrzający jest stosowany na końcu,
przed filtrem usuwającym szum,
ponieważ wyostrzanie podbija istniejący na obrazie szum.
Filtr korekty histogramu kontrastu jest aplikowany na początku,
by poprawić czytelność zdjęcia dla następnych filtrów.
Poniżej dokładniej opisaliśmy działanie każdego etapu korekty.

\newpage

\begin{figure}[H]
    \centering
    \includegraphics[angle=90,width=\linewidth, keepaspectratio]{Photos4/syrenka.jpg}
    \caption{Zdjęcie przed obróbką.}
\end{figure}
\begin{figure}[H]
    \centering
    \includegraphics[width=\linewidth, keepaspectratio]{Photos4/syrenka_PTI.jpg}
    \caption{Zdjęcie poprawione -- metodą adaptacyjną.}
\end{figure}

\vfill
\subsection{Normalizacja histogramu kontrastu}
W pierwszym kroku program podejmuje decyzję czy zaaplikować filtr
poprawy kontrastu. Filtr ten często zwiększa wartość szumu na zdjęciu,
zatem w przypadku gdy normalizacja histogramu nie jest konieczna
zaaplikowanie tego filtru może znacząco pogorszyć jakość zdjęcia.
Krok ten jest pomijany, kiedy odchylenie standardowe jasności
jest większe niż 0,2. Wartość progu została dobrana metodą prób
i błędów tak, aby nasz program działał dobrze dla zdjęć, które
zebraliśmy na potrzeby projektu.

\subsection{Filtr bilateralny}
Zaszumienie zdjęcia definiuje się jako losowe zakłócenia wartości pikseli zdjęcia.

W drugim kroku program aplikuje filtr bilateralny,
którego siła jest oceniana na podstawie pomiaru szumu na zdjęciu.
Wartość szumu ustalana jest aplikując maskę\footnote{\label{przyp1} Źródło użytych operatorów: Immerkær, J. (1996). Fast Noise Variance Estimation. Computer Vision and Image Understanding, doi:10.1006/cviu.1996.0060, strona 300 }
wykrywającą szum:
\[  L = \begin{bmatrix}
        1  & -2 & 1  \\
        -2 & 4  & -2 \\
        1  & -2 & 1
    \end{bmatrix}   \]

Następnie z powstałego obrazu wyliczane jest odchylenie standardowe,
by ustalić jak bardzo obraz jest zaszumiony. Finalnie aplikowany jest
filtr bilateralny o odchyleniu standardowym równym 9-krotności
wartości szumu.

\subsection{Wyostrzenie zdjęcia}
W ostatnim kroku dobierane są parametry do filtru wyostrzającego
na podstawie wartości akutancji.
Aby obliczyć tą metrykę najpierw aplikowany jest operator Laplace'a\textsuperscript{\ref{przyp1}},
którym wydobywamy krawędzie z obrazu:
\[  L = \begin{bmatrix}
        0  & -1 & 0  \\
        -1 & 4  & -1 \\
        0  & -1 & 0
    \end{bmatrix}\]

Następnie obliczana jest średnia wartość po zaaplikowaniu operatora
(im ostrzejsze krawędzie, tym ostrzejszy obraz).
Jeśli ta wartość przekracza 5,0 to filtr nie jest aplikowany,
by nie wprowadzać dodatkowego szumu.
Jeśli filtr wyostrzający jest aplikowany to wartości filtra
Gaussowskiego są obliczane na podstawie wyznaczonych wcześniej metryk:
\begin{itemize}[label={}, leftmargin=*]
    \item $\sigma$ -- odchylenie standardowe
    \item A -- wartość sredniej akutancji
    \item d -- szerokość filtra
    \item
    \item $\sigma = 5,0 - A$
    \item $d = 3 \sigma$
\end{itemize}



\section{ Opis alternatywnych narzędzi do poprawy jakości zdjęć}
W celu zbadania mocnych i słabych stron naszego programu porównamy
go z innymi rozwiązaniami. Na podstawie tego zestawienia ustalimy,
które aspekty naszego narzędzia wymagają poprawy,
a które dają satysfakcjonujące efekty i wybijają nasze rozwiązanie na
tle innych. Niżej opiszemy działanie programów GIMP,
PhotoGlory i Hotpot.ai, które zawrzemy w naszym zestawieniu.

\subsection{GIMP}
GIMP (GNU Image Manipulation Program) to popularne narzędzie Open Source do obróbki zdjęć. Program oferuje szeroką gamę funkcji służących do regulacji jasności i kontrastu, redukcji szumów i usuwania obiektów (np. zagięć) ze zdjęć.

Dostosowanie kontrastu i jasności było bardzo łatwe, a wyniki satysfakcjonujące. GIMP oferuje wiele opcji kontroli cieni, natomiast zauważalny jest brak narzędzi do precyzyjnego zarządzania jasnymi partiami obrazu - rozjaśnianie ciemnych obszarów prowadzi do prześwietlenia jasnych elementów, takich jak niebo na przykładzie zdjęcia syrenki. Program gorzej sobie radzi z poprawą ostrości, brak w nim funkcji do tego dedykowanych.  Jeżeli chodzi o redukcję szumów, GIMP wypada bardzo dobrze. Program posiada wiele zaawansowanych wyspecjalizowanych narzędzi do odszumiania zdjęć.

Wadą programu jest nieintuicyjne rozmieszczenie parametrów, funkcje przydatne przy poprawie jakości zdjęcia wymagają ich poszukiwania w obszernym menu. Liczba opcji też może być przytłaczająca dla początkującego użytkownika. Poprawa jakości zdjęć w programie GIMP również zajmuje bardzo dużo czasu i dla uzyskania dobrych rezultatów wymagane są zaawansowane umiejętności obsługi tego programu.

\begin{figure}[H]
    \centering
    \includegraphics[width=\linewidth, keepaspectratio]{Photos4/zdjecia po poprawie + tablica/syrenka_oryginal.png}
    \caption{Zdjęcie przed obróbką w GIMP-ie.}
\end{figure}
\begin{figure}[H]
    \centering
    \includegraphics[width=\linewidth, keepaspectratio]{Photos4/zdjecia po poprawie + tablica/syrenka_gimp.png}
    \caption{Zdjęcie poprawione -- po obróbce w GIMP-ie.}
\end{figure}

\vfill
\subsection{PhotoGlory}
PhotoGlory jest profesjonalną aplikacją do naprawy zdjęć. Poradziła sobie ona bardzo dobrze z poprawą jasności i kontrastu, zaciemnione wcześniej detale zaczęły być widoczne. Narzędzie to również skutecznie usuwa szumy przy zachowaniu ostrości na zdjęciu.

Aplikacja jest łatwiejsza w obsłudze niż GIMP, posiada mniej narzędzi, bo specjalizuje się konkretnie w naprawie zdjęć, jej cel jest mniej ogólny. Opcje są łatwe do zrozumienia i szczegółowo opisane. PhotoGlory można uznać za narzędzie bardzo intuicyjne w obsłudze. Wadą tego rozwiązania jest jednak ilość czasu poświęcona na naprawę jednego zdjęcia.
\begin{figure}[H]
    \centering
    \includegraphics[width=\linewidth, keepaspectratio]{Photos4/zdjecia po poprawie + tablica/syrenka_oryginal.png}
    \caption{Zdjęcie przed obróbką w PhotoGlory.}
\end{figure}
\begin{figure}[H]
    \centering
    \includegraphics[width=\linewidth, keepaspectratio]{Photos4/zdjecia po poprawie + tablica/syrenka_photoglory.png}
    \caption{Zdjęcie poprawione -- po obróbce w PhotoGlory.}
\end{figure}



\vfill
\subsection{Hotpot.ai}
Hotpot.ai to narzędzie AI udostępniane do użytku online. Na stronie internetowej dostępne są opcje generowania zdjęć i tekstu, a także poprawiania jakości zdjęć. Skorzystaliśmy z narzędzia służącego do naprawy starych fotografii.

Ze wszystkich testowanych przez nas narzędzi to poradziło sobie najlepiej z odszumianiem i poprawianiem ostrości zdjęcia. Na fotografii wynikowej nie widać ziarna, czego inne testowane przez nas algorytmy nie były w stanie tego osiągnąć bez kompletnej utraty ostrości, a tutaj krawędzie nie tylko pozostały ostre, ale też są bardziej uwydatnione

Pomimo poprawy kontrastu zdjęcia główny obiekt na zdjęciu (w tym przypadku warszawska syrenka) nie stał się bardziej widoczny, algorytm nie usunął cienia z ciemnych obszarów i detale pozostają ukryte. Użytkownik nie może dostosować algorytmu do swoich potrzeb, więc wyniki mogą być pod pewnymi względami rozczarowujące. Narzędzie nie było w stanie rozpoznać elementów ciemnych z powodu niedoświetlenia od tych, które były ciemne z natury.

Zaletą narzędzia AI jest wygoda - wystarczy wrzucić zdjęcie na stronę i chwilę zaczekać, nie trzeba dostosowywać żadnych parametrów, proces jest w pełni zautomatyzowany. Wadą tego programu z kolei jest to, że wyniki nie zawsze są powtarzalne. W przypadku AI również często pojawiają się obawy związane z prywatnością i etycznością takich rozwiązań.

\newpage
\begin{figure}[H]
    \centering
    \includegraphics[width=\linewidth, keepaspectratio]{Photos4/zdjecia po poprawie + tablica/syrenka_oryginal.png}
    \caption{Zdjęcie przed obróbką w Hotpot.ai.}
\end{figure}
\begin{figure}[H]
    \centering
    \includegraphics[width=\linewidth, keepaspectratio]{Photos4/zdjecia po poprawie + tablica/syrenka_hotpotai.png}
    \caption{Zdjęcie poprawione -- po obróbce w Hotpot.ai.}
\end{figure}

\newpage
\section{Porównanie zdjęć}
\begin{figure}[H]
    \centering
    \includegraphics[width=\linewidth, keepaspectratio]{Photos4/jamnik_org.jpg}
    \caption{Zdjęcie oryginalne.}
\end{figure}

\newpage

\begin{figure}[H]
    \centering
    \includegraphics[width=\linewidth, keepaspectratio]{Photos4/jamnik_farfocle.png}
    \caption{Zdjęcie z dodanymi farfoclami. }
\end{figure}


Na innych edytowanych przez nas zdjęciach nie pojawił się problem “farfocli”, więc w celu zweryfikowania jak nasz algorytm sobie poradził w walce z farfoclami ręcznie dodaliśmy farfocle na zdjęcie jamnika.

\newpage

\begin{figure}[H]
    \centering
    \includegraphics[width=\linewidth, keepaspectratio]{Photos4/jamnik_PTI.png}
    \caption{Zdjęcie poprawione naszym programem.}
\end{figure}

\newpage

\begin{figure}[H]
    \centering
    \includegraphics[width=\linewidth, keepaspectratio]{Photos4/zdjecia po poprawie + tablica/nowe (dzieci jamnik i pogrzeb)/jamnik_farfocle_photoglory.png}
    \caption{Zdjęcie poprawione za pomocą aplikacji PhotoGlory.}
\end{figure}
\begin{figure}[H]
    \centering
    \includegraphics[width=\linewidth, keepaspectratio]{Photos4/zdjecia po poprawie + tablica/nowe (dzieci jamnik i pogrzeb)/jamnik_farfocle_hotpotai.png}
    \caption{Zdjęcie poprawione w programie Hotpot.ai.}
\end{figure}


Nasze narzędzie, jak już prezentowaliśmy we wcześniejszych raportach, bardzo dobrze radzi sobie z usuwaniem farfocli, jednak w przypadku dużych farfocli na zdjęciu nadal pozostają ślady. W aplikacji PhotoGlory niestety zabrakło narzędzia dedykowanego do usuwania farfocli. Za to Hotpot.ai sobie z tym zadaniem bardzo dobrze poradziło, nawet lepiej niż nasze narzędzie, bo na zdjęciu nie zostały prawie żadne ślady.

Pomimo tego, uważamy, że nasze narzędzie nadal wypada dobrze na tle AI ze względu na obawy użytkowników związane z AI. Także nasz program jest w pełni bezpłatny i umożliwia masowe przerabianie zdjęć w krótkim czasie, gdy AI przeznacza kilka minut na jedno zdjęcie.

Niżej zdjęcia pokazane w celu dalszego porównanie narzędzi, specyfika narzędzi została opisana wyżej, dalsze wnioski z porównania wyciągniemy w następnej sekcji.

\newpage

\begin{figure}[H]
    \centering
    \includegraphics[width=\linewidth, keepaspectratio]{Photos4/dzieci.png}
    \caption{Zdjęcie dzieci -- oryginał.}
\end{figure}
\begin{figure}[H]
    \centering
    \includegraphics[width=\linewidth, keepaspectratio]{Photos4/dzieci_gimp.png}
    \caption{Zdjęcie dzieci -- poprawione w programie Gimp. }
\end{figure}
\begin{figure}[H]
    \centering
    \includegraphics[width=\linewidth, keepaspectratio]{Photos4/zdjecia po poprawie + tablica/nowe (dzieci jamnik i pogrzeb)/dzieci_photoglory.png}
    \caption{Zdjęcie dzieci -- poprawione w programie PhotoGlory.}
\end{figure}
\begin{figure}[H]
    \centering
    \includegraphics[width=\linewidth, keepaspectratio]{Photos4/dzieci_hotpot.png}
    \caption{Zdjęcie dzieci -- poprawione w programie Hotpot.ai.}
\end{figure}
\begin{figure}[H]
    \centering
    \includegraphics[width=\linewidth, keepaspectratio]{Photos4/dzieci_PTI.png}
    \caption{Zdjęcie dzieci -- poprawione w naszym programie. }
\end{figure}


\begin{figure}[H]
    \centering
    \includegraphics[width=\linewidth, keepaspectratio]{Photos4/pogrzeb.png}
    \caption{Zdjęcie pogrzebu -- oryginał.}
\end{figure}
\begin{figure}[H]
    \centering
    \includegraphics[width=\linewidth, keepaspectratio]{Photos4/pogrzeb_gimp.png}
    \caption{Zdjęcie pogrzebu -- poprawione w programie Gimp. }
\end{figure}
\begin{figure}[H]
    \centering
    \includegraphics[width=\linewidth, keepaspectratio]{Photos4/zdjecia po poprawie + tablica/nowe (dzieci jamnik i pogrzeb)/pogrzeb_photoglory.png}
    \caption{Zdjęcie pogrzebu -- poprawione w programie PhotoGlory. }
\end{figure}
\begin{figure}[H]
    \centering
    \includegraphics[width=\linewidth, keepaspectratio]{Photos4/pogrzeb_hotpot.png}
    \caption{Zdjęcie pogrzebu -- poprawione w programie Hotpot. }
\end{figure}
\begin{figure}[H]
    \centering
    \includegraphics[width=\linewidth, keepaspectratio]{Photos4/pogrzeb_PTI.png}
    \caption{Zdjęcie pogrzebu -- poprawione w naszym programie.}
\end{figure}


\section{Podsumowanie}
W porównaniu do innych narzędzi nasz program bardzo dobrze sobie radzi z wyostrzaniem zdjęć i naprawą kontrastu zdjęcia. Za to inne programy, zwłaszcza Hotpot.ai, odszumiają zdjęcia znacznie lepiej od naszego. GIMP to jedyne z narzędzi, które dobrze sobie radzi z usuwaniem zagięć. W kwestii farfocli najlepiej poradziło sobie Hotpot.ai, ale nasz program był niewiele gorszy.

Atutem naszego programu jest prostota użycia i mała ilość czasu, którą trzeba przeznaczyć na obróbkę jednego zdjęcia. Również w żadnym innym programie nie było jednocześnie trybu automatycznego i możliwości ręcznego ustawiania parametrów, my oferujemy oba rozwiązania. Uważamy, że nasze narzędzie, pomimo pewnych ograniczeń, można uznać za konkurencyjne.








\newpage
\section{Dostępność programu}
Na chwilę obecną nasze rozwiązanie jest zaawansowanym programem terminalowym,
kompilowalnym na każdym systemie z dostępną i zainstalowaną biblioteką OpenCV oraz
dowolnym kompilatorem języka C++ zdolnym do kompilacji na tym systemie lub jego pochodnym.

Program dostępny jest na licencji \textit{open source} i jego kod źródłowy można znaleźć na GitHubie
pod adresem:
\begin{center}
    \url{https://github.com/ssk12o/PTI-Foto-Projekt}.
\end{center}
skąd można go łatwo pobrać, skompilować i wykorzystywać dowolnie (choć zgodnie z przeznaczeniem!).




% \newpage
\section{Wykorzystywane narzędzia}
W tej części naszego projektu korzystaliśmy z następujących narzędzi:

\begin{itemize}
    \item Programu i języka Matlab -- do analizy zdjęć;
    \item Języka C++ -- do napisania programu usuwającego artefakty;
    \item Programu VS Code -- do tworzenia, edycji i dokumentacji kodu programu i raportów;
    \item Programu LibreOffice Calc -- do analizy części danych numerycznych;
    \item $\LaTeXe{}$ -- do przygotowania raportu;
    \item Strony Github i programu Git -- do udostępniania, dystrybucji i pracy nad kodem;
    \item 7zip -- do kompresji i dekompresji katalogów zdjęć;
    \item Programów: GIMP, PhotoGlory i Hotpot.ai do edycji zdjęć w celu wykonania porównania narzędzi dostępnych na rynku z naszym programem;
    \item Google Drive -- do udostępniania części dużych plików;
    \item Skanera minilab Noritsu HS-1800 -- do w dalszej części wykonywania wysokiej jakości cyfrowych skanów zdjęć wykonanych techniką analogową;
    \item Aparatów:
          \begin{itemize}
              \item Canon EOS 300 z obiektywem Tamron 28-105mm 1:4-5.6 i kliszą Fomapan 400
              \item Fujifilm FinePix L55 Digital Camera -- Black (12MP, 3x Optical Zoom)
          \end{itemize}
\end{itemize}


\section{Podział obowiązków}
Na tym etapie projektu podzieliśmy się pracą, obowiązkami i zadaniami w następujący sposób:

\begin{itemize}
    \item Bartosz Wójcik -- pisanie raportu.
    \item Natalia Szymańska -- porównanie programu z dostępnymi narzędziami -- opis programów i różnic pomiędzy nimi. % hehe dostarczanie alkoholu członkom zespołu
    \item Katarzyna Szwed -- korekta i pisanie raportu.
    \item Aleksandra Wójcik -- porównanie programu z dostępnymi narzędziami -- poprawianie jakości zdjęć za pomocą tych narzędzi.
    \item Karol Sęk -- tworzenie i rozwijanie algorytmu, poprawianie funkcjonalności programu.
    \item Patrycja Szałajko -- porównanie programu z dostępnymi narzędziami -- opis programów i różnic pomiędzy nimi.
    \item Michał Juszkiewicz -- tworzenie i rozwijanie algorytmu, poprawianie funkcjonalności programu.
    \item Filip Sajko -- implementacja raportu w \LaTeXe{}.
\end{itemize}


\section{Bibliografia}

\begin{enumerate}
    \item Immerkær, J. (1996). Fast Noise Variance Estimation. Computer Vision and Image Understanding, doi:10.1006/cviu.1996.0060.
\end{enumerate}





































\newpage

\part{}
\thispagestyle{empty}

\begin{figure}[h]
    \centering
    \includegraphics[width=1\textwidth]{wspolne_dla_wszystkich/logo_uczelni.png}
\end{figure}


\begin{center}
    {\LARGE \textbf{Poprawa jakości skanów zdjęć wykonanych techniką analogową
        }} \\[0.3cm]
    {\large \textbf{Raport V}} \\[0.2cm]
    \textit{projekt realizowany pod opieką prof. dr hab. inż. Artura Przelaskowskiego}

\end{center}

\begin{figure}[h]
    \centering
    \includegraphics[width=1\textwidth]{wspolne_dla_wszystkich/logo_projektu.png}
\end{figure}

\vfill
\begin{abstract}
    Raport 5 projektu poprawy jakości cyfrowych skanów zdjęć wykonanych techniką analogową przez grupę nr 9 (wtorkową z godziny 18)
    w składzie:  Bartosz Wójcik, Katarzyna Szwed, Natalia Szymańska,
    Patrycja Szałajko, Aleksandra Wójcik, Karol Sęk, Michał Juszkiewicz, Filip Sajko.

    W tym etapie przeprowadziliśmy eksperyment w celu weryfikacji użyteczności naszego programu - poprosiliśmy dwóch użytkowników o zdjęcia i zebraliśmy ich opinie na temat działania naszego narzędzia.
\end{abstract}


\newpage


\section{Korekta}
W związku z uwagami postanowiliśmy w ramach korekty do poprzednich raportów doprecyzować opis algorytmów zaimplementowanych w naszym projekcie.


\subsection{Filtracja filtrem Gausowsskim}
Operacja wykonuje splot zdjęcia z odpowiednim filtrem opartym na rozkładzie gaussowskim
zależnym od parametrów podanych przez użytkownika. Celem tej operacji jest zredukowanie
szumu zdjęcia.
Operacja przyjmuje dwa argumenty: sigma i rozmiar maski. \newline

Wartość sigmy wpływa na siłę rozmycia zdjęcia. Czym większa sigma, tym wartość funkcji w
punkcie $(0,0)$ jest bliższa $0$. Dlatego rozmycie jest silniejsze dla większych wartości sigma.
Rozmiar maski wyznacza wielkość obszaru w pobliżu piksela na którym wykonywany jest splot.
W przedziale $[-3 \sigma, 3\sigma]$ znajduję się $99.73\%$ objętości
rozkładu. Dla masek o rozmiarze większym niż $3\sigma$ wartości oddalone od centrum będą miały
bardzo małe i nieznaczne wartości. Dlatego rozmiar maski nie powinien przekraczać $3\sigma$.
Oczywiście rozmiar musi być liczbą nieparzystą. \newline

Ostatecznie rozmiar maski i sigma powinny być dobrane w zależności do zdjęcia, zbyt duży
rozmiar i wartość sigmy powodują utratę detali zdjęcia, a mała maska lub mała sigma nieznacznie
zmniejszą szum zdjęcia. \newline

Wady metody:
\begin{itemize}
    \item Ryzyko utraty detali.
    \item Trudność w doborze parametrów (są bardzo zależne od zdjęcia, dla każdego powinny być inne).
\end{itemize}

\subsection{Filtracja filtrem bilateralnym}
Operacja działa podobnie do wyżej opisanego filtru gaussowskiego. Wykonuje na zdjęciu splot z
filtrem gaussowskim, różni się tym, że dla każdego piksela w masce jego wartość jest mnożona
przez odpowiednią wagę. Dzięki takiemu działaniu metoda usuwa szum minimalnie rozmywając
krawędzie zdjęcia.

Waga zależna jest od różnicy pomiędzy wartościami pikseli w centrum i otoczeniu.

Waga dla danego piskela jest wyrażona \footnote{Wzór pochodzi z pracy dostępnej pod linkiem: \url{https://www.ipol.im/pub/art/2011/bcm_nlm/article.pdf} }:
\[w(i,j,k,l) = \exp \left(-\frac{(i-k)^2+(j-l)^2}{2\sigma^2}- \frac{||I(i,j) - I(k,l)||^2}{2\sigma^2}\right)\]
Gdzie:
\begin{itemize}
    \item $i,j$ -- współrzędne piksela w centrum
    \item $k,l$ -- współrzędne piksela w otoczeniu, dla którego liczona jest wag
    \item $\sigma$ -- parametr podany przez użytkownika, standardowe odchylenie rozkładu gaussowskiego (patrz filtr gaussowski)
    \item $I(x,y)$ -- funkcja zwracająca wartość piksela na danych współrzędnych x i y. Oczywiście dla (i,j) = (k,l) waga = 1. Operacja przyjmuje dwa argumenty: sigma i rozmiar maski, powinny one być dobrane zgodnie z sugestiami opisanymi w (filtr gaussowski).
\end{itemize}
\newpage


Zalety metody wobec filtru gaussowskiego:
\begin{itemize}
    \item Mniejsze ryzyko utraty krawędzi i detali zdjęcia.
\end{itemize}

Wady metody wobec filtru gaussowskiego:
\begin{itemize}
    \item Zdjęcia zawierające zbyt dużo detali i krawędzi będą nieczułe na działanie filtra.
    \item Zdjęcia przetworzone tym filtrem często wyglądają `kreskówkowo'.
\end{itemize}


\subsection{Usuwanie szumu uśrednianiem pikseli}
Operacja polega na obliczeniu nowej wartości dla danego piskela, która powinna mieć
zredukowany szum. Jeśli dla danego piksela znajdziemy $9$ pikseli o bliskiej wartości i obliczymy
ich średnią to odchylenie standardowe szumu powinno być zredukowane $3$-krotnie. Operacja
poszukuje w zdjęciu jak najwięcej bliskich wartościowo pikseli i liczy ich średnią ważoną.

Nowa wartość piksela jest obliczana wzorem\footnote{Wzór pochodzi z pracy dostępnej pod linkiem: \url{https://www.ipol.im/pub/art/2011/bcm_nlm/article.pdf} }:
\[ \hat{u}_i(p) = \frac{1}{C(p)} \sum_{q \in B(p, r)} u_i(q) \, w(p, q)\]
\[ C(p) = \sum_{q \in B(p,q)}^{}w(p,q) \]
\[ w(p,q) = \exp(-\frac{\max (d^2-2\sigma^2, 0.0)}{h^2})\]

Gdzie:
\begin{itemize}
    \item $B(p,r)$ -- otoczenie piksela p w promieniu r, r -- ustalone na $21$\footnote{czym większe otoczenie tym teoretycznie efektywność filtru byłaby lepsza, niestety już dla $r=21$ operacja zajmuje bardzo dużo czasu, dla większego okna byłaby jeszcze wolniejsza.}.
    \item $u_i(q)$ -- funkcja zwracająca wartość piksela q.
    \item $d(p, q)$ -- odległość euklidesowa pomiędzy pikselami p a q.
    \item $\sigma$ -- odchylenie standardowe.
    \item $h$ -- parametr filtrowania.
\end{itemize}

Operacja przyjmuje jeden argument: $h$ -- parametr filtrowania.
Wartość $h$ wpływa na wartości sigma użyta we wzorze.
Parametr spełnia zależność $h = k\sigma$. Wartość $k$ dopasowywana jest dynamicznie. Czym
większa odległość pomiędzy pikselami tym wartość $k$ jest mniejsza, sigma więc rośnie.
Czym większa wartość $h$ tym filtr będzie silniejszy, ale zwiększa się też ryzyko utraty detali.
Dobrą wartością początkową jest $15.0$\footnote{Wyznaczona empirycznie, jest to tylko sugestia.}, następnie w zależności od zdjęcia należy ją zwiększyć lub
zmniejszyć. \newline



Wady metody:
\begin{itemize}
    \item Ryzyko utraty detali.
    \item Czasochłonność.
    \item Niejednolita efektywność, dla niektórych zdjęć zmiana jest niezazauważalna.

\end{itemize}
Zalety metody wobec filtru gaussowskiego i bilateralnego:
\begin{itemize}
    \item Lepszy stosunek redukcji szumu do rozmazania zdjęcia.
\end{itemize}




\subsection{Wyostrzanie maską wyostrzającą (unsharp masking)}
Operacja ma na celu wyostrzenie krawędzi i detali zdjęcia. Mając ostre zdjęcie możemy je
rozmazać stosując rozmycie gaussowskie, można więc założyć, że wykonując operacje odwrotną do
rozmazania gaussowskiego, efekt będzie odwrotny do rozmycia, wyostrzenie. Na tej zasadzie
polega metoda unsharp masking. Odejmuje od wartości pikseli oryginalnego zdjęcia wartości
pikseli po rozmyciu.

Operacja przyjmuje dwa argumenty: sigma i rozmiar maski, analogicznie do (filtru gaussowskiego).
Teoretycznie większe rozmazanie zdjęcia powinno bardziej wyostrzyć zdjęcie, należy więc dobierać
większe wartości sigma (przynajmniej $>1$). \newline

Wady metody:
\begin{itemize}
    \item Trudność w doborze parametrów (są bardzo zależne od zdjęcia, dla każdego powinny być inne).
    \item Właściwie operacja rozmazywania zdjęcia jest nieodwracalna, metoda może mieć różne efekty.
\end{itemize}





\newpage
\section{Metodyka eksperymentu}

Głównym celem naszego programu było podniesienie jakości cyfrowych skanów starych fotografii analogowych. Szczególnie zależało nam na poprawie tych ujęć, które charakteryzują się niepoprawną ekspozycją -- zawierających obszary silnie prześwietlone, gdzie detale zostały praktycznie `wypalone' przez nadmiar światła lub fragmenty głęboko niedoświetlone, w których szczegóły giną w jednolitej czerni. Zmierzyliśmy się również z „farfoclami", czyli artefaktami powstałymi w procesie wywoływania i skanowania zdjęć. Postanowiliśmy przetestować nasz program w realnych warunkach, dlatego poprosiliśmy dwóch ochotników o użycie go na ich własnych materiałach fotograficznych. \newline

Pierwszy użytkownik jest pasjonatem fotografii analogowej. Nasze narzędzie wydało mu się ciekawy i z chęcią przyłączył się do testowania, gdyż jak każdy początkujący fotograf-amator miał wiele sytuacji, gdy na żywo zaobserwował interesującą scenę, którą chciał uchwycić w obiektywie, lecz nie miał wystarczająco dobrych warunków, aby wykonać dobre jakościowo zdjęcia. Widać to na zdjęciach pociągu, gdzie podziemia utrudniły uchwycenie sceny w odpowiednim świetle. Nasz użytkownik dzięki naszemu programowi miał nadzieję na uratowanie mocno niedoświetlonego zdjęcia i nasz program dawał perspektywy nie tylko na poprawę dostarczonych zdjęć, ale również dawał perspektywę na rozszerzenie zakresu fotografowanych scenerii o miejsca, które wcześniej wydawały się niedostępne ze względu na brak odpowiednich warunków. \newline

Celem drugiej użytkowniczki jest naprawa starych zdjęć rodzinnych. Nasz program mógłby jej pomóc w obróbce fotografii z albumów, zwłaszcza, że dopiero zaczyna z obróbką zdjęć i jeszcze nie umie dobrze obsługiwać zaawansowanych programów, np. Gimpa. Też jej się bardzo nie podoba czas, który musi spędzić na obróbkę jednego zdjęcia przy użyciu innych programów, podczas gdy przeróbka jednego obrazu w naszym programie trwa kilka minut. \newline

Skany zdjęć użytkowników zostały przerobione za pomocą metody adaptacyjnej dostępnej w naszym programie. Następnie użytkownicy ocenili jej efekty w skali od -3 do 3 i wyrazili swoje opinie na temat naszego narzędzia.


% \begin{figure}[h!]
%     \centering
%     \begin{subfigure}[b]{0.49\textwidth}
%         \centering
%         \includegraphics[width=\textwidth]{Photos5/}
%         \caption{Przed.}
%         %     \end{subfigure}
%     \hfill
%     \begin{subfigure}[b]{0.49\textwidth}
%         \centering
%         \includegraphics[width=\textwidth]{Photos5/}
%         \caption{Po działaniu programu.}
%         %     \end{subfigure}
%     \caption{Zdjęcie budynku. Ocena użytkownika:    }
%     % \end{figure}



\newpage
\section{Test}
\subsection{Użytkownik 1}

\subsubsection{Zdjęcia}

\begin{figure}[h!]
    \centering
    \begin{subfigure}[b]{0.49\textwidth}
        \centering
        \includegraphics[width=\textwidth]{Photos5/użytkownik 1 (bartek)/budynek1.jpg}
        \caption{Przed.}
    \end{subfigure}
    \hfill
    \begin{subfigure}[b]{0.49\textwidth}
        \centering
        \includegraphics[width=\textwidth]{Photos5/użytkownik 1 (bartek)/budynek1_po.jpg}
        \caption{Po działaniu programu.}
    \end{subfigure}
    \caption{Zdjęcie budynku. Ocena użytkownika: 0}
\end{figure}
Zdjęcie oryginalne przed obróbką jest niedoświetlone i trudne jest rozróżnienie detali. Po obróbce jest bardziej czytelne, ale nie jest estetycznie akceptowalne. Poziom zakłóceń i szumów powoduje, że efekt finalny nie jest przyjemny dla oka. Zarysowania na zdjęciu są nadal widoczne po obróbce, jednak wtapiają się w inne szumy. Zważając na to, że zdjęcie wejściowe jest niskiej jakości, efekt wyjściowy w innych programach również nie byłby dużo lepszy.


\begin{figure}[h!]
    \centering
    \begin{subfigure}[b]{0.49\textwidth}
        \centering
        \includegraphics[width=\textwidth]{Photos5/użytkownik 1 (bartek)/pociag.jpg}
        \caption{Przed.}
    \end{subfigure}
    \hfill
    \begin{subfigure}[b]{0.49\textwidth}
        \centering
        \includegraphics[width=\textwidth]{Photos5/użytkownik 1 (bartek)/pociag_po1.jpg}
        \caption{Po działaniu programu.}
    \end{subfigure}
    \caption{Zdjęcie pociągu. Ocena użytkownika: 1    }
\end{figure}
Efekt obróbki tego zdjęcia jest bardzo podobny jak do poprzedniego. Szum znacząco pogarsza odbiór zdjęcia. Filtr działałby lepiej, gdyby wybierał obszary zdjęcia, w których da się coś poprawić. Front pociągu w oryginale jest niedoświetlony do tego stopnia, że filtr jedynie powoduje wprowadzenie dużego poziomu szumów. Zdjęcie zyskuje na czytelności, filtr wydobył detale, które ciężko było odróżnić na oryginale.




\begin{figure}[h!]
    \centering
    \begin{subfigure}[b]{0.49\textwidth}
        \centering
        \includegraphics[angle=-90, width=\textwidth]{Photos5/użytkownik 1 (bartek)/drzwi.jpg}
        \caption{Przed.}
    \end{subfigure}
    \hfill
    \begin{subfigure}[b]{0.49\textwidth}
        \centering
        \includegraphics[width=\textwidth]{Photos5/użytkownik 1 (bartek)/drzwi_po1.jpg}
        \caption{Po działaniu programu.}
    \end{subfigure}
    \caption{Zdjęcie drzwi. Ocena użytkownika:  2  }
\end{figure}
Ze wszystkich zdjęć danych do obróbki, jakość po filtracji jest najwyższa dla tego zdjęcia. Zdjęcie jest wyraźnie jaśniejsze oraz poziom szumu jest akceptowalny. Efekty w cieniach są lepsze, niż na innych zdjęciach. Niestety zdjęcie po obróbce wydaje się prześwietlone w jasnych punktach. Filtr zaaplikowany na tym zdjęciu jest zbyt ekstremalny, zdjęcie potrzebowało tylko lekkiej korekty histogramu, by poszerzyć rozpiętość tonalną. \newline

Filtr działał podobnie dla reszty zdjęć, stąd brak komentarza dla efektów dla każdego z nich.
\newpage




\begin{figure}[h!]
    \centering
    \begin{subfigure}[b]{0.49\textwidth}
        \centering
        \includegraphics[angle=90, width=\textwidth]{Photos5/użytkownik 1 (bartek)/cos.jpg}
        \caption{Przed.}
    \end{subfigure}
    \hfill
    \begin{subfigure}[b]{0.49\textwidth}
        \centering
        \includegraphics[angle=90, width=\textwidth]{Photos5/użytkownik 1 (bartek)/cos_po1.jpg}
        \caption{Po działaniu programu.}
    \end{subfigure}
    \caption{Zdjęcie czegoś. Ocena użytkownika: 2    }
\end{figure}

\begin{figure}[h!]
    \centering
    \begin{subfigure}[b]{0.49\textwidth}
        \centering
        \includegraphics[width=\textwidth]{Photos5/użytkownik 1 (bartek)/peron1.jpg}
        \caption{Przed.}
    \end{subfigure}
    \hfill
    \begin{subfigure}[b]{0.49\textwidth}
        \centering
        \includegraphics[width=\textwidth]{Photos5/użytkownik 1 (bartek)/peron1_po.jpg}
        \caption{Po działaniu programu.}
    \end{subfigure}
    \caption{Zdjęcie peronu. Ocena użytkownika: 0   }
\end{figure}



\begin{figure}[h!]
    \centering
    \begin{subfigure}[b]{0.49\textwidth}
        \centering
        \includegraphics[width=\textwidth]{Photos5/użytkownik 1 (bartek)/schody.jpg}
        \caption{Przed.}
    \end{subfigure}
    \hfill
    \begin{subfigure}[b]{0.49\textwidth}
        \centering
        \includegraphics[width=\textwidth]{Photos5/użytkownik 1 (bartek)/schody_po1.jpg}
        \caption{Po działaniu programu.}
    \end{subfigure}
    \caption{Zdjęcie schodów. Ocena użytkownika:  1  }
\end{figure}


\begin{figure}[h!]
    \centering
    \begin{subfigure}[b]{0.49\textwidth}
        \centering
        \includegraphics[width=\textwidth]{Photos5/użytkownik 1 (bartek)/wejscie.jpg}
        \caption{Przed.}
    \end{subfigure}
    \hfill
    \begin{subfigure}[b]{0.49\textwidth}
        \centering
        \includegraphics[width=\textwidth]{Photos5/użytkownik 1 (bartek)/wejscie_po1.jpg}
        \caption{Po działaniu programu.}
    \end{subfigure}
    \caption{Zdjęcie wejścia. Ocena użytkownika:  2  }
\end{figure}



\newpage
\subsubsection{Komentarz ogólny}
Filtr dobrze radzi sobie z polepszeniem czytelności zdjęcia, jednak nie nadaje się do obróbki w celach estetycznych. Szum powodowany przez filtr jest zbyt duży, filtr nie selekcjonuje obszarów do poprawy. Skutkuje to zaszumionymi cieniami oraz prześwietlonymi punktami, które poprawy nie wymagały. Program przy odpowiednim rozwoju mógłby poradzić sobie lepiej z ekstremalnymi przypadkami.





\newpage
\subsection{Użytkowniczka 2}

\subsubsection{Zdjęcia}


\begin{figure}[h!]
    \centering
    \begin{subfigure}[b]{0.49\textwidth}
        \centering
        \includegraphics[angle=-90, width=\textwidth]{Photos5/użytkownik 2 (ola)/choinka.png}
        \caption{Przed.}
    \end{subfigure}
    \hfill
    \begin{subfigure}[b]{0.49\textwidth}
        \centering
        \includegraphics[angle=-90, width=\textwidth]{Photos5/użytkownik 2 (ola)/choinka_po1.png}
        \caption{Po działaniu programu.}
    \end{subfigure}
    \caption{Zdjęcie dzieci i choinki. Ocena użytkownika:  1.25  (spodziewałam się lepszych efektów, ale zdjecie jest ostrzejsze i widać zmianę)}
\end{figure}
Widać różnicę. Postacie są bardziej widoczne przez zwiększony kontrast, jednak wciąż jestem w stanie dostrzec lekkie zagięcia oryginalnego zdjęcia, a także farfocel, który na oryginalnym zdjęciu najbardziej przykuwał moją uwagę i chciałam się go pozbyć (ten na ręce starszej dziewczynki). Zdjęcie jest ostrzejsze, jednak kolory w miarę są zbalansowane i nie widać szumu.

\begin{figure}[h!]
    \centering
    \begin{subfigure}[b]{0.49\textwidth}
        \centering
        \includegraphics[width=\textwidth]{Photos5/użytkownik 2 (ola)/idk.jpg}
        \caption{Przed.}
    \end{subfigure}
    \hfill
    \begin{subfigure}[b]{0.49\textwidth}
        \centering
        \includegraphics[width=\textwidth]{Photos5/użytkownik 2 (ola)/idk_po.jpg}
        \caption{Po działaniu programu.}
    \end{subfigure}
    \caption{Zdjęcie ogrodu. Ocena użytkownika:  2 (niby mała różnica, ale mnie zadowoliła i spełniła moje oczekiwania, zdecydowanie ułatwi mi dalszą pracę nad tym zdjęciem)  }
\end{figure}
Na zdjęciu widać poprawę. Nie widzę szumu, zniknęły farfocle, które przeszkadzały mi na poprzednim zdjęciu. Jako ewentualny drobny minus zauważyłam, że część białych spodenek mężczyzny została uznana za farfocel i przez to pojawiła się na nich maleńka ciemna plama. Też fajnie jakby zniknęło zagięcie, lecz już poradzę sobie z tym sama. Ważne, że małe farfocle usunęło - bardzo ułatwi mi to pracę nad samodzielnym dalszym retuszem tego zdjęcia.


\begin{figure}[h!]
    \centering
    \begin{subfigure}[b]{0.49\textwidth}
        \centering
        \includegraphics[angle=90, width=\textwidth]{Photos5/użytkownik 2 (ola)/pradziadkowieuwu.jpg}
        \caption{Przed.}
    \end{subfigure}
    \hfill
    \begin{subfigure}[b]{0.49\textwidth}
        \centering
        \includegraphics[angle=90, width=\textwidth]{Photos5/użytkownik 2 (ola)/pradziadkowieuwu_po.jpg}
        \caption{Po działaniu programu.}
    \end{subfigure}
    \caption{Zdjęcie pradziadków. Ocena użytkownika: 1.75  (małe różnice, ale dają wystarczający efekt by można było to uznać za poprawę) }
\end{figure}
Widać różnicę, ładnie wyrównaliście kolory w tle przez co to zdjęcie wygląda na gładsze. Dodatkowo zniknął farfocel na środku, który przykuwał wcześniej moją uwagę, więc nowe zdjęcie zyskało na jakości. Na minus jest to, że wciąż widać jakieś plamy na ubraniach postaci na dole.




\subsubsection{Komentarz ogólny}
Narzędzie zdecydowanie poprawia jakość zdjęcia. W miarę dobrze radzi sobie z usuwaniem farfocli i wyostrzaniem. Widzę potencjał na rozbudowę programu by można było np. usuwać zagięcia. Na razie jest to dobre narzędzie, z którego można korzystać przy edycji zdjęć - najpierw przepuścić przez program, żeby pozbyć się farfocli i wyostrzyć, a potem samemu dokończyć edycję zdjęć aby uzyskać jak najlepszy efekt. Na pewno to narzędzie bardzo skraca czas spędzony na przeróbce zdjęcia, często nad zdjęciami siedzę godzinami w gimpie i męczę się z edycją… Zdecydowanie otrzymanie takiego wstępnie przerobionego zdjęcia przyspieszyłoby mi robotę.



\newpage
\section{Wnioski}
Przetworzone zdjęcia użytkowników oraz ich oceny ukazały wady oraz zalety działania naszego programu.\newline

Użytkownik 1 przyznał, że pomimo większej ilości widocznych detali dzięki rozjaśnieniu i wyostrzeniu zdjęcia nie podobają mu się efekty uzyskane przez nasze narzędzie. Skrytykował fakt, że wszystkie obszary, nawet te, które tego nie potrzebowały, zostały rozjaśnione. Uważa też, że działanie filtru było zbyt mocne - dobrane wartości filtrów są zbyt ekstremalne dla zdjęć, które wymagają lekkiej korekty. Również nie podobała mu się ilość szumu dodanego przez filtr, co jest głównym powodem dlaczego Użytkownik 1 stwierdził, że nie skorzystałby z naszego narzędzia. Najważniejsza dla niego jest estetyka zdjęcia i nie może uznać takiego poziomu szumu za akceptowalny wynik. \newline

Użytkowniczce 2 dużo bardziej spodobał się nasz program. Uznała, że na przerobionych zdjęciach widać poprawę - są one ostrzejsze, kolory są bardziej wyrównane i, co najważniejsze, zniknęły uporczywe farfocle. Zauważyła też wady w naszym rozwiązaniu - nasz program nie usuwa wszystkich farfocli lub czasem usuwa te fragmenty, które nimi nie są. Przeszkadzało jej też to, że nasz program nie radzi sobie dobrze z usuwaniem zagięć. Użytkowniczka stwierdziła, że nasze narzędzie dobrze radzi sobie z tymi aspektami obróbki zdjęcia, które sprawiały jej największą trudność. Nasze narzędzie pozwoliło jej oszczędzić trochę czasu pomimo tego, że wynik uzyskany przez nasz program nie jest tym ostatecznym i zdjęcia nadal wymagają dalszej obróbki w innym programie.

\begin{table}[H]
    \centering
    \begin{tabular}{|c|c|}
        \hline
        Średnia      & Wartość \\ \hline
        Użytkownik 1 & 1.14    \\ \hline
        Użytkownik 2 & 1.67    \\ \hline
        Sumaryczna   & 1.40    \\ \hline
    \end{tabular}
    \caption{Średnie oceny działania programu.}
\end{table}

Tym co zdecydowanie wymaga poprawy w naszym programie jest usuwanie szumu i ilość szumu wprowadzana do zdjęcia przez nasze algorytmy. Również w naszym programie zabrakło opcji zlokalizowania filtru - wszystkie w naszym narzędziu działa globalnie, na całym zdjęciu, nie tylko na jego fragmencie. Dobrym pomysłem byłoby również dodanie algorytmu wyspecjalizowanego w usuwaniu zagięć. Ogółem nasze narzędzie poradziło sobie dobrze z mało ekstremalnymi przypadkami - im ciemniejsze/mniej ostre zdjęcie tym więcej szumu wprowadzają nasze filtry. Nasz program również bardzo dobrze sobie radzi z wydobywaniem ukrytych detali. Znaleźliśmy również zastosowanie dla naszego programu - umożliwia on sprawne i skuteczne usunięcie farfocli oszczędzając użytkownikowi dużo czasu.


















\newpage
\section{Dostępność programu}
Na chwilę obecną nasze rozwiązanie jest zaawansowanym programem terminalowym,
kompilowalnym na każdym systemie z dostępną i zainstalowaną biblioteką OpenCV oraz
dowolnym kompilatorem języka C++ zdolnym do kompilacji na tym systemie lub jego pochodnym.

Program dostępny jest na licencji \textit{open source} i jego kod źródłowy można znaleźć na GitHubie
pod adresem:
\begin{center}
    \url{https://github.com/ssk12o/PTI-Foto-Projekt}.
\end{center}
skąd można go łatwo pobrać, skompilować i wykorzystywać dowolnie (choć zgodnie z przeznaczeniem!).




% \newpage
\section{Wykorzystywane narzędzia}
W tej części naszego projektu korzystaliśmy z następujących narzędzi:

\begin{itemize}
    \item Programu i języka Matlab -- do analizy zdjęć;
    \item Języka C++ -- do napisania programu usuwającego artefakty;
    \item Programu VS Code -- do tworzenia, edycji i dokumentacji kodu programu i raportów;
    \item Programu LibreOffice Calc -- do analizy części danych numerycznych;
    \item $\LaTeXe{}$ -- do przygotowania raportu;
    \item Strony Github i programu Git -- do udostępniania, dystrybucji i pracy nad kodem;
    \item 7zip -- do kompresji i dekompresji katalogów zdjęć;
    \item Programów: GIMP, PhotoGlory i Hotpot.ai do edycji zdjęć w celu wykonania porównania narzędzi dostępnych na rynku z naszym programem;
    \item Google Drive -- do udostępniania części dużych plików;
    \item Skanera minilab Noritsu HS-1800 -- do w dalszej części wykonywania wysokiej jakości cyfrowych skanów zdjęć wykonanych techniką analogową;
    \item Aparatów:
          \begin{itemize}
              \item Canon EOS 300 z obiektywem Tamron 28-105mm 1:4-5.6 i kliszą Fomapan 400
              \item Fujifilm FinePix L55 Digital Camera -- Black (12MP, 3x Optical Zoom)
          \end{itemize}
\end{itemize}


\section{Podział obowiązków}
Na tym etapie projektu podzieliśmy się pracą, obowiązkami i zadaniami w następujący sposób:

\begin{itemize}
    \item Bartosz Wójcik -- zebranie komentarzy użytkowników.
    \item Natalia Szymańska -- opisanie eksperymentu i wyciągniętych wniosków, korekta. % hehe dostarczanie alkoholu członkom zespołu
    \item Katarzyna Szwed -- opisanie eksperymentu i wyciągniętych wniosków, korekta.
    \item Aleksandra Wójcik -- zebranie komentarzy użytkowników.
    \item Karol Sęk -- dokładny opis algorytmów.
    \item Patrycja Szałajko -- opisanie eksperymentu i wyciągniętych wniosków, korekta.
    \item Michał Juszkiewicz -- dokładny opis algorytmów.
    \item Filip Sajko -- implementacja raportu w \LaTeX{}.
\end{itemize}












\end{document}