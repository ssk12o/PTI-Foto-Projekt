\documentclass[]{mwart}

\usepackage{polski}
\usepackage[utf8]{inputenc}

\usepackage{amsthm}
\usepackage{amsmath}
\usepackage{amssymb}

\usepackage{mdframed}
\usepackage{hyperref}
\usepackage[draft%      % dla obrazkow zakomentowac draft
]{graphicx}  
\usepackage{url}
\usepackage{enumitem}
\usepackage{verbatim}



\usepackage{caption}    
\usepackage{float}      



\usepackage{fancyhdr}
\pagestyle{fancy}
\fancyhf{}

\fancyhead[L]{\includegraphics[height=0.666cm]{wspolne_dla_wszystkich/logo_projektu.png}}
\fancyhead[C]{\textit{Poprawa jakości zdjęć}}
\fancyhead[R]{\includegraphics[height=0.9cm]{wspolne_dla_wszystkich/logo_uczelni.png}}
\fancyfoot[C]{\thepage}

\setlength{\headheight}{20pt}  



\usepackage{listings}
\usepackage{xcolor} 




\begin{document}
\thispagestyle{empty}

\begin{figure}[h]
    \centering
    \includegraphics[width=1\textwidth]{wspolne_dla_wszystkich/logo_uczelni.png}
\end{figure}


\begin{center}
    {\LARGE \textbf{Poprawa jakości skanów zdjęć wykonanych techniką analogową
        }} \\[0.3cm]
    {\large \textbf{Raport II}} \\[0.2cm]
    \textit{projekt realizowany pod opieką prof. dr hab. inż. Artura Przelaskowskiego}

\end{center}

\begin{figure}[h]
    \centering
    \includegraphics[width=1\textwidth]{wspolne_dla_wszystkich/logo_projektu.png}
\end{figure}

\vfill
\begin{abstract}
    Raport 3 projektu poprawy jakości cyfrowych skanów zdjęć wykonanych techniką analogową przez grupę nr 9 (wtorkową z godziny 18)
    w składzie:  Bartosz Wójcik, Katarzyna Szwed, Natalia Szymańska,
    Patrycja Szałajko, Aleksandra Wójcik, Karol Sęk, Michał Juszkiewicz, Filip Sajko.

    W tym raporcie opiszemy nasze działania prowadzące do poprawy działania programu i opiszemy poprawki.
    Zajmiemy się ponadto ekstensywnym testowaniem działania naszego programu i wyciagnieciem wniosków na temat jego działania i optymalnych ustawień.
\end{abstract}

\newpage
\tableofcontents{}

\newpage

% ADD KOREKTA
\section{Korekta do raportu 2}
\begin{itemize}
    \item //to do
\end{itemize}


\section{Rozszerzanie działalności programu}
\subsection{Słowem wstępu...}
Głównym celem naszego programu jest coś, co można nazwać
ogólnie\footnote{konkretnie nasz cel zdefiniowaliśmy w poprzednim raporcie.}
poprawą jakość skanów zdjęć analogowych. Trzeba jednak pamiętać,
że najważniejszym ogniwem każdego problemy jest człowiek, którego
problem ten dotyczy -- i któremu chce się pomóc oferując dane rozwiązania. \newline

Tak więc łatwo zauważyć, że kluczowym elementem każdego programu jest dobrze wykonany
interfejs użytkownika. W ostatecznym rozrachunku przecież gorszy
jest przecież program z nawet najlepszym algorytmem skrytym pod płaszczem źle zaprojektowanego,
nieczytelnego i nieprzyjaznego interfejsu od którego użytkownik się odbije
i koniec końców z usług którego nie skorzysta. \newline

Rozumiejąc tę potrzebę wraz z optymalizacją i rozszerzaniem funkcjonalności
naszego programu stworzyliśmy interfejs użytkownika i zaczęliśmy % tja... na pewno
tworzyć szczegółową instrukcję obsługi naszego programu.

\subsection{Menu programu}
Na chwilę obecną główne menu naszego programu oferuje następujące opcje:

\begin{table}[h]
    \centering
    \begin{tabular}{|c|l|}
        \hline
        Lp. & Opcja                                  \\ \hline
        1   & usuwanie farfocli                      \\ \hline
        2   & normalizacja histogramu jasności       \\ \hline
        3   & filtracja filtrem bilateralnym         \\ \hline
        4   & filtracja filtrem gassowskim           \\ \hline
        5   & usuwaniem szumu -- uśrednianie pikseli \\ \hline
        6   & wyostrzenie -- maska wyostrzająca      \\ \hline
    \end{tabular}
    \caption{Opcje główne naszego programu dostane dla użytkownika}
\end{table}

Teraz zajmiemy się szczegółowym omówieniem wymienionych w powyżej tabeli opcji. \newpage






\section{Usuwanie farfocli                      }
\section{Normalizacja histogramu jasności       }
\section{Filtracja filtrem bilateralnym         }
\section{Filtracja filtrem gassowskim           }
\section{Usuwaniem szumu -- uśrednianie pikseli }
\section{Wyostrzenie -- maska wyostrzająca      }



\section{Eksperymenty -- studiom ciekawych przypadków}
Tu będą wyniki naszych eksperymentów z ciekawych ustawień dla różnych zdjęć.







\newpage

\section{Dalsze kroki}
Pracujemy obecnie nad stworzeniem bardziej przystępnego interfejsu okienkowego, a przy tym
planujemy lepszą integrację poszczególnych elementów programu.
W dalszym ciągu poszukujemy metod optymalizacji jego działania.

\section{Dostępność programu}
Na chwilę obecną nasze rozwiązanie jest zaawansowanym programem terminalowym,
działającym na każdym systemie z dostępną i zainstalowaną biblioteką OpenCV oraz
dowolnym kompilatorem języka C++ zdolnym do kompilacji na systemie tym lub pochodnym.

Program dostępny jest na licencji \textit{open source} i jego kod źródłowy można znaleźć na GitHubie
pod adresem:
\begin{center}
    \url{https://github.com/ssk12o/PTI-Foto-Projekt}.
\end{center}
skąd można go łatwo pobrać, skompilować i wykorzystywać dowolnie (choć zgodnie z przeznaczeniem!).




\newpage
\section{Wykorzystywane narzędzia}
W tej części naszego projektu korzystaliśmy z następujących narzędzi:
\begin{itemize}
    \item Programu i języka Matlab -- do analizy zdjęć;
    \item Języka C++ -- do napisania programu usuwającego artefakty;
    \item Programu VS Code -- do tworzenia, edycji i dokumentacji kodu programu i raportów;
    \item Programu LibreOffice Calc -- do analizy części danych numerycznych;
    \item $\LaTeXe{}$ -- do przygotowania raportu;
    \item Strony Github i programu Git -- do udostępniania, dystrybucji i pracy nad kodem;
    \item 7zip -- do kompresji zdjęć;
    \item Google Drive -- do udostępniania części dużych plików;
    \item Skanera minilab Noritsu HS-1800 -- do w dalszej części wykonywania wysokiej jakości cyfrowych skanów zdjęć wykonanych techniką analogową;
    \item Aparatów:
          \begin{itemize}
              \item Canon EOS 300 z obiektywem Tamron 28-105mm 1:4-5.6 i kliszą Fomapan 400
              \item Fujifilm FinePix L55 Digital Camera -- Black (12MP, 3x Optical Zoom)
          \end{itemize}
\end{itemize}


\section{Podział obowiązków}
Na tym etapie projektu podzieliśmy się pracą, obowiązkami i zadaniami w następujący sposób:
\begin{itemize}
    \item Bartosz Wójcik -- wykonywanie, skanowanie i analiza zdjęć; research.
    \item Katarzyna Szwed -- korekta raportu; analiza zdjęć i działania programu.
    \item Natalia Szymańska -- pisanie raportu.
    \item Patrycja Szałajko -- zarządzanie pracą zespołu, kontakt.
    \item Aleksandra Wójcik -- skanowanie zdjęć rodzinnych w celu polepszenia ich jakości w końcowych etapach projektu.
    \item Karol Sęk -- tworzenie, analizowanie i pisanie algorytmu.
    \item Michał Juszkiewicz -- tworzenie, analizowanie i pisanie algorytmu.
    \item Filip Sajko -- pisanie raportu, implementacja w \LaTeX{}.
\end{itemize}


%                       Wersja alternatywna podziału obowiązków
% \section{Podział obowiązków}
% Na tym etapie projektu podzieliśmy się pracą, obowiązkami i zadaniami w następujący sposób:
% 
% \begin{table}[h!]
%     \centering
%     \renewcommand{\arraystretch}{1.3}
%     \begin{tabular}{|p{3cm}|p{7.5cm}|} \hline
%         \textbf{Imię i nazwisko} & \textbf{Zakres obowiązków}                                                               \\ \hline \hline
%         Bartosz Wójcik           & Wykonywanie, skanowanie i analiza zdjęć; opieka merytoryczna.                            \\ \hline
%         Katarzyna Szwed          & Tworzenie, analizowanie i pisanie algorytmu; korekta raportu.                            \\ \hline
%         Natalia Szymańska        & Pisanie raportu.                                                                         \\ \hline
%         Patrycja Szałajko        & Zarządzanie pracą zespołu, kontakt z mediami.                                            \\ \hline
%         Aleksandra Wójcik        & Skanowanie zdjęć rodzinnych w celu polepszenia ich jakości w końcowych etapach projektu. \\ \hline
%         Karol Sęk                & Tworzenie, analizowanie i pisanie algorytmu.                                             \\ \hline
%         Michał Juszkiewicz       & Tworzenie, analizowanie i pisanie algorytmu.                                             \\ \hline
%         Filip Sajko              & Pisanie raportu, implementacja w \LaTeX{}.                                               \\ \hline
%     \end{tabular}
%     \caption{Podział obowiązków w zespole projektowym.}
% \end{table}




\end{document}