\documentclass[]{mwart}

\usepackage{polski}
\usepackage[utf8]{inputenc}

\usepackage{amsthm}
\usepackage{amsmath}
\usepackage{amssymb}

\usepackage{mdframed}
\usepackage{hyperref}
\usepackage[%draft%      % dla obrazkow zakomentowac draft
]{graphicx}  
\usepackage{url}
\usepackage{enumitem}
\usepackage{verbatim}
\usepackage{caption}


    
\usepackage{float}      



\usepackage{fancyhdr}
\pagestyle{fancy}
\fancyhf{}

\fancyhead[L]{\includegraphics[height=0.666cm]{wspolne_dla_wszystkich/logo_projektu.png}}
\fancyhead[C]{\textit{Poprawa jakości zdjęć}}
\fancyhead[R]{\includegraphics[height=0.9cm]{wspolne_dla_wszystkich/logo_uczelni.png}}
\fancyfoot[C]{\thepage}

\setlength{\headheight}{20pt}  



\usepackage{listings}
\usepackage{xcolor} 




\begin{document}
\thispagestyle{empty}

\begin{figure}[h]
    \centering
    \includegraphics[width=1\textwidth]{wspolne_dla_wszystkich/logo_uczelni.png}
\end{figure}


\begin{center}
    {\LARGE \textbf{Poprawa jakości skanów zdjęć wykonanych techniką analogową
        }} \\[0.3cm]
    {\large \textbf{Raport IV}} \\[0.2cm]
    \textit{projekt realizowany pod opieką prof. dr hab. inż. Artura Przelaskowskiego}

\end{center}

\begin{figure}[h]
    \centering
    \includegraphics[width=1\textwidth]{wspolne_dla_wszystkich/logo_projektu.png}
\end{figure}

\vfill
\begin{abstract}
    Raport 4 projektu poprawy jakości cyfrowych skanów zdjęć wykonanych techniką analogową przez grupę nr 9 (wtorkową z godziny 18)
    w składzie:  Bartosz Wójcik, Katarzyna Szwed, Natalia Szymańska,
    Patrycja Szałajko, Aleksandra Wójcik, Karol Sęk, Michał Juszkiewicz, Filip Sajko.

    Na tym etapie projektu dodaliśmy tryb automatycznej korekty zdjęcia --
    dodaliśmy presety wartości do poszczególnych filtrów oraz tryb automatycznego doboru filtrów.
    Będziemy również porównywać nasz program z popularnymi narzędziami do obróbki zdjęć - Gimp, PhotoGlory oraz HotpotAI.
    Na podstawie zestawienia wyników otrzymanych za pomocą tych trzech narzędzi ustalimy wady i zalety naszego rozwiązania.
\end{abstract}

\newpage
\tableofcontents{}

\newpage

% \section{Korekta do raportu 3}
% \begin{itemize}
%     \item
%     \item
%     \item
%     \item
% \end{itemize}


\section{Tryb automatyczny (presety)}
Każdy z filtrów dostępnych w naszym programie oblicza domyślne
wartości dla danego zdjęcia na podstawie kontrastu, ostrości i
poziomu zaszumienia zdjęcia.
W programie nadal pozostaje opcja ręcznego ustawiania wartości
dla niestandardowych zdjęć.

Wprowadziliśmy także metodę adaptacyjną, wykorzystuje ona trzy
filtry do poprawy zdjęcia:
\begin{center}
    Normalizacja histogramu jasności $\to$ Filtr bilateralny $\to$ Wyostrzenie zdjęcia
\end{center}

Dokładne działanie pojedynczych filtrów opisaliśmy we wcześniejszych raportach.

Program dobiera odpowiednie wartości dla filtrów (lub który filtr zaaplikować)
na podstawie 3 metryk - odchylenie standardowe jasności dla filtru
korekty histogramu jasności, odchylenie standardowe szumu dla
filtru bilateralnego, wartości akutancji zdjęcia dla filtru
poprawy ostrości. Filtr wyostrzający jest stosowany na końcu,
przed filtrem usuwającym szum,
ponieważ wyostrzanie podbija istniejący na obrazie szum.
Filtr korekty histogramu kontrastu jest aplikowany na początku,
by poprawić czytelność zdjęcia dla następnych filtrów.
Poniżej dokładniej opisaliśmy działanie każdego etapu korekty.

\begin{figure}[H]
    \centering
    \includegraphics[angle=90,width=\linewidth, keepaspectratio]{Photos4/syrenka.jpg}
    \caption{Zdjęcie przed obróbką.}
\end{figure}
\begin{figure}[H]
    \centering
    \includegraphics[width=\linewidth, keepaspectratio]{Photos4/syrenka_PTI.jpg}
    \caption{Zdjęcie poprawione -- po obróbce.}
\end{figure}

\subsection{Normalizacja histogramu kontrastu}
W pierwszym kroku program podejmuje decyzję czy zaaplikować filtr
poprawy kontrastu. Filtr ten często zwiększa wartość szumu na zdjęciu,
zatem w przypadku gdy normalizacja histogramu nie jest konieczna
zaaplikowanie tego filtru może znacząco pogorszyć jakość zdjęcia.
Krok ten jest pomijany, kiedy odchylenie standardowe jasności
jest większe niż 0,2. Wartość progu została dobrana metodą prób
i błędów tak, aby nasz program działał dobrze dla zdjęć, które
zebraliśmy na potrzeby projektu.

\subsection{Filtr bilateralny}
Zaszumienie zdjęcia definiuje się jako losowe zakłócenia wartości pikseli zdjęcia.

Zaszumienie zdjęcia definiuje się jako losowe
zakłócenia wartości pikseli zdjęcia.
W drugim kroku program aplikuje filtr bilateralny,
którego siła jest oceniana na podstawie pomiaru szumu na zdjęciu.
Wartość szumu ustalana jest aplikując maskę\footnote{\label{przyp1}Immerkær, J. (1996). Fast Noise Variance Estimation. Computer Vision and Image Understanding, doi:10.1006/cviu.1996.0060, strona 300 }
wykrywającą szum:
\[  L = \begin{bmatrix}
        1  & -2 & 1  \\
        -2 & 4  & -2 \\
        1  & -2 & 1
    \end{bmatrix}   \]

Następnie z powstałego obrazu wyliczane jest odchylenie standardowe,
by ustalić jak bardzo obraz jest zaszumiony. Finalnie aplikowany jest
filtr bilateralny o odchyleniu standardowym równym 9-krotności
wartości szumu.

\subsection{Wyostrzenie zdjęcia}
W ostatnim kroku dobierane są parametry do filtru wyostrzającego
na podstawie wartości akutancji.
Aby obliczyć tą metrykę najpierw aplikowany jest operator Laplace'a,
którym wydobywamy\textsuperscript{\ref{przyp1}}
krawędzie z obrazu:
\[  L = \begin{bmatrix}
        0  & -1 & 0  \\
        -1 & 4  & -1 \\
        0  & -1 & 0
    \end{bmatrix}\]

Następnie obliczana jest średnia wartość po zaaplikowaniu operatora
(im ostrzejsze krawędzie, tym ostrzejszy obraz).
Jeśli ta wartość przekracza 5,0 to filtr nie jest aplikowany,
by nie wprowadzać dodatkowego szumu.
Jeśli filtr wyostrzający jest aplikowany to wartości filtra
Gaussowskiego są obliczane na podstawie wyznaczonych wcześniej metryk:
\begin{itemize}[label={}, leftmargin=*]
    \item $\sigma$ -- odchylenie standardowe
    \item A -- wartość sredniej akutancji
    \item d -- szerokość filtra
    \item
    \item $\sigma = 5,0 - A$
    \item $d = 3 \sigma$
\end{itemize}



\section{ Opis alternatywnych narzędzi do poprawy jakości zdjęć}
W celu zbadania mocnych i słabych stron naszego programu porównamy
go z innymi rozwiązaniami. Na podstawie tego zestawienia ustalimy,
które aspekty naszego narzędzia wymagają poprawy,
a które dają satysfakcjonujące efekty i wybijają nasze rozwiązanie na
tle innych. Niżej opiszemy działanie programów GIMP,
PhotoGlory i Hotpot.ai, które zawrzemy w naszym zestawieniu.

\subsection{GIMP}
GIMP (GNU Image Manipulation Program) to popularne narzędzie Open Source do obróbki zdjęć. Program oferuje szeroką gamę funkcji służących do regulacji jasności i kontrastu, redukcji szumów i usuwania obiektów (np. zagięć) ze zdjęć.

Dostosowanie kontrastu i jasności było bardzo łatwe, a wyniki satysfakcjonujące. GIMP oferuje wiele opcji kontroli cieni, natomiast zauważalny jest brak narzędzi do precyzyjnego zarządzania jasnymi partiami obrazu - rozjaśnianie ciemnych obszarów prowadzi do prześwietlenia jasnych elementów, takich jak niebo na przykładzie zdjęcia syrenki. Program gorzej sobie radzi z poprawą ostrości, brak w nim funkcji do tego dedykowanych.  Jeżeli chodzi o redukcję szumów, GIMP wypada bardzo dobrze. Program posiada wiele zaawansowanych wyspecjalizowanych narzędzi do odszumiania zdjęć.

Wadą programu jest nieintuicyjne rozmieszczenie parametrów, funkcje przydatne przy poprawie jakości zdjęcia wymagają ich poszukiwania w obszernym menu. Liczba opcji też może być przytłaczająca dla początkującego użytkownika. Poprawa jakości zdjęć w programie GIMP również zajmuje bardzo dużo czasu i dla uzyskania dobrych rezultatów wymagane są zaawansowane umiejętności obsługi tego programu.

\begin{figure}[H]
    \centering
    \includegraphics[width=\linewidth, keepaspectratio]{Photos4/zdjecia po poprawie + tablica/syrenka_oryginal.png}
    \caption{Zdjęcie przed obróbką w GIMP-ie.}
\end{figure}
\begin{figure}[H]
    \centering
    \includegraphics[width=\linewidth, keepaspectratio]{Photos4/zdjecia po poprawie + tablica/syrenka_gimp.png}
    \caption{Zdjęcie poprawione -- po obróbce w GIMP-ie.}
\end{figure}

\subsection{PhotoGlory}
PhotoGlory jest profesjonalną aplikacją do naprawy zdjęć. Poradziła sobie ona bardzo dobrze z poprawą jasności i kontrastu, zaciemnione wcześniej detale zaczęły być widoczne. Narzędzie to również skutecznie usuwa szumy przy zachowaniu ostrości na zdjęciu.

Aplikacja jest łatwiejsza w obsłudze niż GIMP, posiada mniej narzędzi, bo specjalizuje się konkretnie w naprawie zdjęć, jej cel jest mniej ogólny. Opcje są łatwe do zrozumienia i szczegółowo opisane. PhotoGlory można uznać za narzędzie bardzo intuicyjne w obsłudze. Wadą tego rozwiązania jest jednak ilość czasu poświęcona na naprawę jednego zdjęcia.
\begin{figure}[H]
    \centering
    \includegraphics[width=\linewidth, keepaspectratio]{Photos4/zdjecia po poprawie + tablica/syrenka_oryginal.png}
    \caption{Zdjęcie przed obróbką w PhotoGlory.}
\end{figure}
\begin{figure}[H]
    \centering
    \includegraphics[width=\linewidth, keepaspectratio]{Photos4/zdjecia po poprawie + tablica/syrenka_photoglory.png}
    \caption{Zdjęcie poprawione -- po obróbce w PhotoGlory.}
\end{figure}




\subsection{ Hotpot.ai}
Hotpot.ai to narzędzie AI udostępniane do użytku online. Na stronie internetowej dostępne są opcje generowania zdjęć i tekstu, a także poprawiania jakości zdjęć. Skorzystaliśmy z narzędzia służącego do naprawy starych fotografii.

Ze wszystkich testowanych przez nas narzędzi to poradziło sobie najlepiej z odszumianiem i poprawianiem ostrości zdjęcia. Na fotografii wynikowej nie widać ziarna, czego inne testowane przez nas algorytmy nie były w stanie tego osiągnąć bez kompletnej utraty ostrości, a tutaj krawędzie nie tylko pozostały ostre, ale też są bardziej uwydatnione

Pomimo poprawy kontrastu zdjęcia główny obiekt na zdjęciu (w tym przypadku warszawska syrenka) nie stał się bardziej widoczny, algorytm nie usunął cienia z ciemnych obszarów i detale pozostają ukryte. Użytkownik nie może dostosować algorytmu do swoich potrzeb, więc wyniki mogą być pod pewnymi względami rozczarowujące. Narzędzie nie było w stanie rozpoznać elementów ciemnych z powodu niedoświetlenia od tych, które były ciemne z natury.

Zaletą narzędzia AI jest wygoda - wystarczy wrzucić zdjęcie na stronę i chwilę zaczekać, nie trzeba dostosowywać żadnych parametrów, proces jest w pełni zautomatyzowany. Wadą tego programu z kolei jest to, że wyniki nie zawsze są powtarzalne. W przypadku AI również często pojawiają się obawy związane z prywatnością i etycznością takich rozwiązań.

\begin{figure}[H]
    \centering
    \includegraphics[width=\linewidth, keepaspectratio]{Photos4/zdjecia po poprawie + tablica/syrenka_oryginal.png}
    \caption{Zdjęcie przed obróbką w Hotpot.ai.}
\end{figure}
\begin{figure}[H]
    \centering
    \includegraphics[width=\linewidth, keepaspectratio]{Photos4/zdjecia po poprawie + tablica/syrenka_hotpotai.png}
    \caption{Zdjęcie poprawione -- po obróbce w Hotpot.ai.}
\end{figure}

\newpage
\section{Pokazane wyniki zdjęć po przeróbce}

\section{Podsumowanie}








\newpage
\section{Dostępność programu}
Na chwilę obecną nasze rozwiązanie jest zaawansowanym programem terminalowym,
kompilowalnym na każdym systemie z dostępną i zainstalowaną biblioteką OpenCV oraz
dowolnym kompilatorem języka C++ zdolnym do kompilacji na tym systemie lub jego pochodnym.

Program dostępny jest na licencji \textit{open source} i jego kod źródłowy można znaleźć na GitHubie
pod adresem:
\begin{center}
    \url{https://github.com/ssk12o/PTI-Foto-Projekt}.
\end{center}
skąd można go łatwo pobrać, skompilować i wykorzystywać dowolnie (choć zgodnie z przeznaczeniem!).




% \newpage
\section{Wykorzystywane narzędzia}
W tej części naszego projektu korzystaliśmy z następujących narzędzi:

\begin{itemize}
    \item Programu i języka Matlab -- do analizy zdjęć;
    \item Języka C++ -- do napisania programu usuwającego artefakty;
    \item Programu VS Code -- do tworzenia, edycji i dokumentacji kodu programu i raportów;
    \item Programu LibreOffice Calc -- do analizy części danych numerycznych;
    \item $\LaTeXe{}$ -- do przygotowania raportu;
    \item Strony Github i programu Git -- do udostępniania, dystrybucji i pracy nad kodem;
    \item 7zip -- do kompresji i dekompresji katalogów zdjęć;
    \item Programów: GIMP, PhotoGlory i Hotpot.ai do edycji zdjęć w celu wykonania porównania narzędzi dostępnych na rynku z naszym programem;
    \item Google Drive -- do udostępniania części dużych plików;
    \item Skanera minilab Noritsu HS-1800 -- do w dalszej części wykonywania wysokiej jakości cyfrowych skanów zdjęć wykonanych techniką analogową;
    \item Aparatów:
          \begin{itemize}
              \item Canon EOS 300 z obiektywem Tamron 28-105mm 1:4-5.6 i kliszą Fomapan 400
              \item Fujifilm FinePix L55 Digital Camera -- Black (12MP, 3x Optical Zoom)
          \end{itemize}
\end{itemize}


\section{Podział obowiązków}
Na tym etapie projektu podzieliśmy się pracą, obowiązkami i zadaniami w następujący sposób:

\begin{itemize}
    \item Bartosz Wójcik -- pisanie raportu.
    \item Natalia Szymańska -- porównanie programu z dostępnymi narzędziami -- opis programów i różnic pomiędzy nimi. % hehe dostarczanie alkoholu członkom zespołu
    \item Katarzyna Szwed -- korekta i pisanie raportu.
    \item Aleksandra Wójcik -- porównanie programu z dostępnymi narzędziami -- poprawianie jakości zdjęć za pomocą tych narzędzi.
    \item Karol Sęk -- tworzenie i rozwijanie algorytmu, poprawianie funkcjonalności programu.
    \item Patrycja Szałajko -- porównanie programu z dostępnymi narzędziami -- opis programów i różnic pomiędzy nimi.
    \item Michał Juszkiewicz -- tworzenie i rozwijanie algorytmu, poprawianie funkcjonalności programu.
    \item Filip Sajko -- implementacja raportu w \LaTeXe{}.
\end{itemize}


\section{Bibliografia}
W tej części raoportu kożystaliśmy z następujących prac naukowych i źródeł akademickich:
\begin{itemize}
    \item Immerkær, J. (1996). Fast Noise Variance Estimation. Computer Vision and Image Understanding, doi:10.1006/cviu.1996.0060.
\end{itemize}


\end{document}