\documentclass[]{mwart}

\usepackage{polski}
\usepackage[utf8]{inputenc}

\usepackage{amsthm}
\usepackage{amsmath}
\usepackage{amssymb}

\usepackage{mdframed}
\usepackage{hyperref}
\usepackage[%draft%      % dla obrazkow zakomentowac draft
]{graphicx}  
\usepackage{url}
\usepackage{enumitem}
\usepackage{verbatim}
\usepackage{caption}


    
\usepackage{float}      



\usepackage{fancyhdr}
\pagestyle{fancy}
\fancyhf{}

\fancyhead[L]{\includegraphics[height=0.666cm]{wspolne_dla_wszystkich/logo_projektu.png}}
\fancyhead[C]{\textit{Poprawa jakości zdjęć}}
\fancyhead[R]{\includegraphics[height=0.9cm]{wspolne_dla_wszystkich/logo_uczelni.png}}
\fancyfoot[C]{\thepage}

\setlength{\headheight}{20pt}  



\usepackage{listings}
\usepackage{xcolor} 




\begin{document}
\thispagestyle{empty}

\begin{figure}[h]
    \centering
    \includegraphics[width=1\textwidth]{wspolne_dla_wszystkich/logo_uczelni.png}
\end{figure}


\begin{center}
    {\LARGE \textbf{Poprawa jakości skanów zdjęć wykonanych techniką analogową
        }} \\[0.3cm]
    {\large \textbf{Raport III}} \\[0.2cm]
    \textit{projekt realizowany pod opieką prof. dr hab. inż. Artura Przelaskowskiego}

\end{center}

\begin{figure}[h]
    \centering
    \includegraphics[width=1\textwidth]{wspolne_dla_wszystkich/logo_projektu.png}
\end{figure}

\vfill
\begin{abstract}
    Raport 4 projektu poprawy jakości cyfrowych skanów zdjęć wykonanych techniką analogową przez grupę nr 9 (wtorkową z godziny 18)
    w składzie:  Bartosz Wójcik, Katarzyna Szwed, Natalia Szymańska,
    Patrycja Szałajko, Aleksandra Wójcik, Karol Sęk, Michał Juszkiewicz, Filip Sajko.

    W tym raporcie...
\end{abstract}

\newpage
\tableofcontents{}

\newpage

\section{Korekta do raportu 2}
\begin{itemize}
    \item
    \item
    \item
    \item
\end{itemize}




\section{Dostępność programu}
Na chwilę obecną nasze rozwiązanie jest zaawansowanym programem terminalowym,
kompilowalnym na każdym systemie z dostępną i zainstalowaną biblioteką OpenCV oraz
dowolnym kompilatorem języka C++ zdolnym do kompilacji na tym systemie lub jego pochodnym.

Program dostępny jest na licencji \textit{open source} i jego kod źródłowy można znaleźć na GitHubie
pod adresem:
\begin{center}
    \url{https://github.com/ssk12o/PTI-Foto-Projekt}.
\end{center}
skąd można go łatwo pobrać, skompilować i wykorzystywać dowolnie (choć zgodnie z przeznaczeniem!).




\newpage
\section{Wykorzystywane narzędzia}
W tej części naszego projektu korzystaliśmy z następujących narzędzi:

\[\mathbb{TO\_BE\_UPDATED}\] Trzeba dodać nowe rzeczy, w tym programy testowane

\begin{itemize}
    \item Programu i języka Matlab -- do analizy zdjęć;
    \item Języka C++ -- do napisania programu usuwającego artefakty;
    \item Programu VS Code -- do tworzenia, edycji i dokumentacji kodu programu i raportów;
    \item Programu LibreOffice Calc -- do analizy części danych numerycznych;
    \item $\LaTeXe{}$ -- do przygotowania raportu;
    \item Strony Github i programu Git -- do udostępniania, dystrybucji i pracy nad kodem;
    \item 7zip -- do kompresji zdjęć;
    \item Google Drive -- do udostępniania części dużych plików;
    \item Skanera minilab Noritsu HS-1800 -- do w dalszej części wykonywania wysokiej jakości cyfrowych skanów zdjęć wykonanych techniką analogową;
    \item Aparatów:
          \begin{itemize}
              \item Canon EOS 300 z obiektywem Tamron 28-105mm 1:4-5.6 i kliszą Fomapan 400
              \item Fujifilm FinePix L55 Digital Camera -- Black (12MP, 3x Optical Zoom)
          \end{itemize}
\end{itemize}


\section{Podział obowiązków}
Na tym etapie projektu podzieliśmy się pracą, obowiązkami i zadaniami w następujący sposób:

\[\mathbb{TO\_BE\_UPDATED}\]
\begin{itemize}
    \item Bartosz Wójcik -- wykonywanie, skanowanie i analiza zdjęć; research.
    \item Katarzyna Szwed -- korekta raportu; analiza zdjęć i działania programu.
    \item Natalia Szymańska -- pisanie raportu.
    \item Patrycja Szałajko -- zarządzanie pracą zespołu, kontakt.
    \item Aleksandra Wójcik -- skanowanie zdjęć rodzinnych w celu polepszenia ich jakości w końcowych etapach projektu.
    \item Karol Sęk -- tworzenie, analizowanie i pisanie algorytmu.
    \item Michał Juszkiewicz -- tworzenie, analizowanie i pisanie algorytmu.
    \item Filip Sajko -- pisanie raportu, implementacja w \LaTeXe{}.
\end{itemize}



\end{document}